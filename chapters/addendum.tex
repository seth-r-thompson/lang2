\setchapterpreamble[u]{\margintoc}
\chapter{Culture}
\section{Metaphors}
Some common conceptual metaphors in \langname{} are given below; most things are also noted in the dictionary. 

\paragraph{\sc wisdom is a tree}
Trees are an important part of \langname{} culture, representing the emotional and intellectual lifespan of a person. Old, tall deciduous trees are a symbol of widsom and maturity; as a plant may \rz{ossat} “grow,” so too may a person \rz{ossat} “become wiser.” Young children attend a \rz{pebar} “garden” for primary education. A student might lament that a former mentor \rz{tąvat} “becomes senile (\emph{lit.} loses leaves),” or that an elder has \rz{} “thinning branches.”

\paragraph{\sc politics is a shore}
The coastline is a common conceptual metaphor for things involving governance. A common idiom for kingdom or state is \rz{taspa u tesa} “sea and shore,” its citizens are \rz{adahę́s} “sands,” its influence \rz{almanizar} “waves.” A tribune might \rz{kirǫyam} “investigate (\emph{lit.} delve into)” an issue, or a \rz{cunvarą} “columnist (\emph{lit.} seagull)” may discuss it.

\paragraph{\sc body parts are skills}
Traditional \langname{} culturual and spiritual beliefs posit that the person is divided into two parts, a perfect body and an imperfect mind. As such many body parts are sacred to certain skills or traits---\rz{hora} “wrist” represents craftmanship; \rz{ahka} “foot” represents wisdom or learnedness; \rz{kamc} “back” represents labor; \rz{makra} “chest” represents responsibility. Someone with poor skill may be \rz{orra} “feeble” or \rz{???} “uncoordinated,” while someone unresponsibile may be \rz{oca} “thin.”

\setchapterpreamble[u]{\margintoc}
\chapter{Registers}
\section{Poetry}
The common structure of a classical \langname{} poem has nine lines, with alternating pairs of short lines and long lines. Short lines are two trochees between two amphibrachs, and long lines are four amphibrachs. The middle three lines of the verse share a central, interrelated metaphor, and the first and last lines feature repitition. Common poetic features also include alliteration across lines and rhyming of stress syllables. Syncope is often used to fit the meter.
