\setchapterpreamble[u]{\margintoc}
\chapter{Constructions}

\section{Copula}
There are three grammatical copula in \langname{}: \rz{ot}, \rz{pit}, and \rz{het}. All three are intransitive verbs, meaning they only take a single argument, the subject; there are no true predicative complements. Instead, the property associated with the subject is typically treated as a modifier in its phrase, no different from any other noun phrase.

\subsection{\rz{ot}}
\rz{Ot} is the most generic copula, used to link the subject to an adjective or another noun.

\paragraph{Adjective-like senses}
Adjective-like senses are syntactically identical to a noun phrase with a  modifier. Since both entities are part of a single phrase, they agree in evidence. 

\begin{examples}
    \baarucols2
    \ex
        \script Sapan ąns oci.
        \bits sapan ąns oci
        \gloss {the dog}:KN wet:KN is
        \tr The dog is wet. 
    \ex \label{ex:ot_appositive}
        \script sec Vęcin oci.
        \bits sec vęcin oci
        \gloss he mercenary is
        \tr He's cruel.
\end{examples}

This construction is also common for nouns that have different senses when used appositively, like (\ref{ex:ot_appositive}).

\paragraph{Noun-like senses}
Since constructions like (\ref{ex:ot_appositive}) skew towards an adjective-like reading, there is an alternative construction that entails a noun-like reading.

\begin{examples}
    \ex \label{ex:ot_disambiguation}
        \script sec Oci z-vęci.
        \bits sec oci ez=vęci
        \gloss he is in=mercenary
        \tr He's a mercenary.
\end{examples}

Syntactically, \rz{z-vęci} is an adjunct. \rz{Ez} is the most common preposition to introduce these adjuncts.

For nouns without a common appositive sense, the default \rz{ot} construction is still used.

\subsection{\rz{pit}}
\rz{Pit} is the possessive copula, used to link the subject with its possessor.

\begin{examples}
    \baarucols2
    \ex
        \script Sapan sej pici.
        \bits sapan sej pici
        \gloss {the dog}:KN his:KN is
        \tr The dog is his. 
    \ex
        \script Sapan m-lalan pici.
        \bits sapan im=lalan pici
        \gloss {the dog} of=auntie is
        \tr The dog is auntie's.
\end{examples}

\subsection{\rz{het}}
\rz{Het} is the locative copula, used to link the subject with a locative modifier, typically a prepositional phrase.

\begin{example}
    \script Sapan h-tesa heci.
    \bits sapan ah=tesa heci
    \gloss {the dog} at=beach is
    \tr The dog is on the beach.
\end{example}