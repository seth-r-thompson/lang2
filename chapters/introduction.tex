\setchapterpreamble[u]{\margintoc}
\chapter{Background}
\section{Origins}
\langname{} is an \emph{a priori} artlang originally conceived to fulfill a speedlang challenge, then modified to handle a relay, then finally molded into its own project. Although I had no clear motivations in mind when beginning it, I found that the confluence of random decisions---the Gleb-generated inventory, challenge stipulations, the early translation choices---made the project into something I really enjoyed.

\section{Goals}
The overarching goal is to create something that is cool to me. The project is meant to be naturalistic, but when naturalism conflicts with aesthetic, aesthetic will be prioritized.

A primary way to grow this language and develop new ideas will be through translation of poetry and scientific journals. I hope that these will push the limits of my syntactical rules while also developing an interesting corpus. I'll also try to use \tsc{5moyd}s and hopefully at some point a journal to expand my ability to speak the language and get an intuitive sense for what constructions it prefers.

As I develop this conlang, my goals will be to explore analytic constructions and syntactic minutiae, write in-depth documentation of how information structure manifests in the language, and produce a robust dictionary and corpus.

\section{Lore}
\langname{} is set in a worldbuilding project I create as a hobby. The language is an isolate spoken on the eastern coast of a peninsula, once the language of city-states and now a common language throughout a number of coastal countries. It is heavily influenced by East Cape, the \emph{de jure} language of the peninsula, and other languages of trade. The world the speakers know is analogous to our 1930s.

\setchapterpreamble[u]{\margintoc}
\chapter{Overview}
\section{Typology}
\langname{} is largely head-initial: verb phrases are broadly verb-object, noun phrases noun-modifier, and prepositional phrases preposition-complement. This pattern holds rather consistently throughout the language, but there are some exceptions, most notably a prominent focus-fronting system.\marginnote{Focus fronting is discussed further in §\ref{sec:fronting}.}

There are five total word classes in \langname{}. Most content words are nouns and verbs, but there is a small, closed set of adjectives. There are also two types of function words, prepositions and particles.

\section{Syntax}
A basic intransitive clause in \langname{} is composed of a subject followed by a predicate.  

\begin{examples}
    \baarucols2
    \ex
        \script Kanyi hec.
        \bits Kanyi hec
        \gloss Kanyi {is there}
        \tr It's Kanyi.
    \ex
        \script Mevan tahęci.
        \bits mevan tahęci
        \gloss {the cat} stretches
        \tr The cat's stretching.
\end{examples}

In a neutral transitive clause, the object follows the verb. Any other dependents are also typically after the object, as in (\ref{ex:word_order_adjunct}).

\begin{examples}
    \ex
        \script Nassoin kąsteci kagesan.
        \bits nassoin kąsteci kagesan
        \gloss {the king} leads {the army}
        \tr The king leads the army.
    \ex \label{ex:word_order_adjunct}
        \script Lalan kiraysi sapan h-tesan.
        \bits lalan kiraysi sapan ah=tesan
        \gloss auntie {looks for} {the dog} on={the beach}
        \tr Auntie's looking for the dog on the beach.
\end{examples}

In some clauses that are more grammatically or informationally marked, other orders are found. Such orders are discussed further in §\ref{sec:fronting}.

\section{Morphology}
Nouns inflect for number and evidence.\marginnote{The pragmatics of nominal evidentiality are discussed in §\ref{sec:evidence}.} Although the generic singular is the citation form, the direct evidence form is the most common in spoken or written corpuses.

% Direct evidence is marked with the suffix -\rz{n}. Indirect evidence is marked by shifting stress to the final syllable. The plural is marked with the suffix -\rz{zr}.

\begin{table}[h]
    \begin{subtable}[t]{.48\textwidth}
        \centering
        \caption{Evidence}
        \begin{tabular}{cc}
        \toprule
            \bf \sc kn & -\rz{i} \\
            \bf \sc un & -\rz{◌́} \\
        \bottomrule
        \end{tabular}%
    \end{subtable}
    \begin{subtable}[t]{.48\textwidth}
        \centering
        \caption{Number}
        \begin{tabular}{cc}
            \toprule
            \bf \sc sg & - \\
            \bf \sc pl & -\rz{zr} \\
            \bottomrule
        \end{tabular}%
    \end{subtable}
    \caption{Noun inflection}
\end{table}

Modifiers agree with their head nouns for evidence. This includes adjectives (like \rz{ócaa}), appositive nouns (like \rz{mulkas}), and possessive pronouns (like \rz{rat}).

\begin{example}
    \script mavá ocaá mulkás rát
    \bits mavá ocaá mulkás rát
    \gloss cat:UN tall:UN fur:UN me:UN
    \tr my skinny fuzzy cat (I haven't met yet)
\end{example}

Verbs inflect for person of subjects and objects. They also agree with the evidence of the subject.\marginnote[*-1]{The rules governing verbal marking are discussed in §\ref{sec:person_agreement} and §\ref{sec:evidence_agreement}.}

\begin{table}[h]
    \begin{subtable}[t]{.3\textwidth}%
        \centering
        \caption{Person (transitive)}
        \begin{tabular}{cc}
        \toprule
            \bf \sc 1.pt & -\rz{r} \\
            \bf \sc 1.ag & -\rz{rc} \\
            \bf \sc nt & -\rz{z} \\
            \bf \sc cm & -\rz{s} \\
            \bf \sc 1 \& nt & -\rz{ns} \\
            \bf \sc refl & -\rz{k} \\
            \bottomrule
        \end{tabular}%
    \end{subtable}%
    \hfill%
    \begin{subtable}[t]{.3\textwidth}%
        \centering
        \caption{Person (intransitive)}
        \begin{tabular}{cc}
            \toprule
            \bf \sc 1 & -\rz{r} \\
            \bf \sc 2 & -\rz{a} \\
            \bf \sc nt & -\rz{z} \\
            \bf \sc cm & -\rz{s} \\
            \bottomrule
        \end{tabular}%
    \end{subtable}%
    \hfill%
    \begin{subtable}[t]{.3\textwidth}%
        \centering
        \caption{Evidence}
        \begin{tabular}{cc}
            \toprule
            \bf \sc kn & -\rz{i} \\
            \bf \sc un & -\rz{◌́} \\
            \bf \sc un.nt & -\rz{óz} \\
            \bf \sc un.cm & -\rz{és} \\
        \bottomrule
        \end{tabular}%
    \end{subtable}%
    \caption{Verb agreement}
\end{table}

There are also two nonfinite verb forms, the bare infinitive and the participle. The infinitive is more noun-like, while the participle is more adverb-like.

\begin{table}[h] \centering
    \begin{tabular}{cc}
        \toprule
        \bf Bare & - \\
        \bf Ptcpl. & -\rz{tal} \\
        \bottomrule
    \end{tabular}
    \caption{Nonfinite verb inflection}
    \label{tab:inflect_verb_nonfinite}
\end{table}

Both nominal and verbal marking patterns are largely agglunative, but surface forms may not always resemble their lemma due to morphophonological processes.