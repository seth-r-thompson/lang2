\setchapterpreamble[u]{\margintoc}
\chapter{Particles}
Particles are a small but open class of discourse markers that can appear at the beginning of a clause. They appear before other constituents of a clause, including even fronted arguments or adjuncts.

\paragraph{Multiple particles}
Only one particle can appear in a clause. If multiple particles are required, a cleft construction is used.

\begin{example*}
    \script yiz: Otr. vi: Mans pesayrs rat, vi: Mans kirayrs rat?
    \bits  yiz ot-r vi mans pesay-rc rat vi mans kiray-rc rat
    \gloss then be-{1} \tsc{q} who marry-\tsc{1.ag} {1} \tsc{q} who find-\tsc{1.ag} {1}
    \tr Now then---who do I marry, who do I find?
    \smoyd https://www.reddit.com/r/conlangs/comments/p31h1u/1518th_just_used_5_minutes_of_your_day/h8oqlvk/ & 1518
\end{example*}  

Although most particles have discourse or conversational implications, some serve more grammatical function. The two most common types of grammatical particles are \emph{clause flaggers} and \emph{conjunctions}.

\section{Clause flaggers}
Clause-flagging particles mark that the clause that follows is not a declarative sentence in some way.

\subsection{\rz{hes}}
The particle \rz{hes} marks polar questions.

\subsection{\rz{vi}}
The particle \rz{vi} marks content questions.

\subsection{\rz{ǫm}}
The particle \rz{ǫm} marks imperatives.

\section{Conjunctions}
Conjuction particles mark that the clause that follows is somehow subordinate to the clause that precedes it.

\subsection{\rz{kai}}
The particle \rz{kai} marks quoted speech. The preceding clause is typically a speech verb.

\subsection{\rz{tęlr}}
The particle \rz{tęlr} marks a clause that is the result of the clause that precedes it.

\subsection{\rz{yiz}}
The particle \rz{yiz} marks a clause that occurs despite the clause that precedes it.

\subsection{\rz{solr}}
The particle \rz{solr} coordinates two clauses.