\setchapterpreamble[u]{\margintoc}
\chapter{Adpositions}
Adpositions are syntactically bound morphemes that express some relationship, often spacial, between constituents. % However, they are considered words, not affixes, because the stress pattern of the noun they bind to does not shift. % \marginnote{Compare \rz{kąsazr} “soldiers,” marked via affix, to \rz{retus im-kęsa}, “soldier's blade,” marked via adposition.}

Adpositions are a closed class, composed of only six members; finer distinctions can be made with \emph{compound adpositions}, such as \rz{t-kams im} “after”, literally “from the back of.” Although many such constructions are common enough to be lexically set, they are not nearly as ubiquitous as lone prepositions.

\section{True prepositions}
There are five true prepositions in \langname{}. These prepositions exclusively appear as verbal adjuncts or as modifiers of a noun phrase.

\subsection{\rz{im}}
The adposition \rz{im} indicates alienable possession. It can also indicate origin. 

\begin{subexamples}
    \baarucols2
    \ex 
        \script retus m-kęsa
        \bits retus im=kęsa
        \gloss blade of=soldier
        \tr soldier's sword
    \ex
        \script kęsa m-Natra
        \bits kęsa im=Natra
        \gloss soldier of=Natra
        \tr soldier from Natra
\end{subexamples}

\rz{Im} is also a common complementizer.

\subsection{\rz{ez}}
The adposition \rz{ez} conveys location inside an object or large body. It can also be used for composition of manmade objects.

\begin{subexamples}
    \baarucols2
    \ex 
        \script gelaǫ́ z-hels
        \bits fort ez=hels
        \gloss fort in=mountain
        \tr fort in mountains
    \ex
        \script hąna z-iama
        \bits hąna ez=iama
        \gloss coil in=knot
        \tr woven cord
\end{subexamples}

\subsection{\rz{tę}}
The adposition \rz{tę} conveys motion relative to a location, either towards or away from.

It's commonly used as an adjunct that conveys the goal of the verb, as in (\ref{ex:tę_as_goal}).

\begin{example}
    \label{ex:tę_as_goal}
    \script rat Ląitr t-kiray rapręn acoan.
    \bits rat ląitr tę-kiray {rapręn acoan}
    \gloss 1 run to=find {bus}
    \tr I had to run to catch the bus.
\end{example}

\subsection{\rz{ah}}
The adposition \rz{ah} conveys location on the surface of another object. It can also be used for general location.

\subsection{\rz{u}}
The adposition \rz{u} conveys association alongside, or is used as a conjunction between two noun phrases.

\begin{subexamples}
    \baarucols2
    \ex
        \script taspa u-tesa
        \bits taspa u=tesa
        \gloss sea and=shore
        \tr sea and shore
    \ex
        \script lanr u-lár
        \bits lanr u=lár
        \gloss this and=that
        \tr this and that
\end{subexamples}

\rz{U} is more limited in semantic scope than other adpositions and is rarely used in compound prepositions to gain further nuance.

\section{Stranded prepositions} \label{sec:stranded_preps}
Stranded prepositions have a similar syntactic distribution to the true prepositions, but have some key differences. Notably, they do not need to have a verbal or nominal head. In this sense they function somewhat like articles, appearing before nouns with some extra meaning or connotation.

\subsection{\rz{su}}
The prototypical stranded preposition is \rz{su}, which has a similar meaning to \rz{u}.

Like other prepositions, \rz{su} can be used to modify a noun phrase. However, it often appears without any nominal or verbal head. This can be used for emphasis.

\begin{example}
    \script s-kagęsa s-kagę́stapa
    \bits su=kagęsa su=kagę́stapa
    \gloss and=army and=navy
    \tr both army and navy
\end{example}

However, the more common use for \rz{su} is simply linking two or more phrases. It's much more flexible than \rz{u} in this regard; for example, it can take prepositional phrases as complements.

\begin{example}
    \script hąna su z-iama z-ecma
    \bits hąna su ez=iama ez=ecma
    \gloss coil and in=knot in=wool
    \tr cord made from wool and fiber
\end{example}

\subsection{Others}
Some compound prepositions are formed with \rz{su}. The most common are \rz{ezzu}, used for associative plurals, and \rz{occu}, used for some quotative constructions.

\setchapterpreamble[u]{\margintoc}
\chapter{Particles}
Particles are a small but open class of discourse markers that can appear at the beginning of a clause. They appear before other constituents of a clause, including even fronted arguments or adjuncts.

\paragraph{Multiple particles}
Only one particle can appear in a clause. If multiple particles are required, a cleft construction is used.

\begin{example*}
    \script yiz: Otr. vi: Mans pesayrs rat, vi: Mans kirayrs rat?
    \bits  yiz ot-r vi mans pesay-rc rat vi mans kiray-rc rat
    \gloss then be-{1} \tsc{q} who marry-\tsc{1.ag} {1} \tsc{q} who find-\tsc{1.ag} {1}
    \tr Now then---who do I marry, who do I find?
    \smoyd https://www.reddit.com/r/conlangs/comments/p31h1u/1518th_just_used_5_minutes_of_your_day/h8oqlvk/ & 1518
\end{example*}  

Although most particles have discourse or conversational implications, some serve more grammatical function. The two most common types of grammatical particles are \emph{clause flaggers} and \emph{conjunctions}.

\section{Clause flaggers}
Clause-flagging particles mark that the clause that follows is not a declarative sentence in some way.

\subsection{\rz{hes}}
The particle \rz{hes} marks polar questions.

\subsection{\rz{vi}}
The particle \rz{vi} marks content questions.

\subsection{\rz{ǫm}}
The particle \rz{ǫm} marks imperatives.

\section{Conjunctions}
Conjuction particles mark that the clause that follows is somehow subordinate to the clause that precedes it.

\subsection{\rz{kai}}
The particle \rz{kai} marks quoted speech. The preceding clause is typically a speech verb.

\subsection{\rz{tęlr}}
The particle \rz{tęlr} marks a clause that is the result of the clause that precedes it.

\subsection{\rz{yiz}}
The particle \rz{yiz} marks a clause that occurs despite the clause that precedes it.

\subsection{\rz{solr}}
The particle \rz{solr} coordinates two clauses.