\setchapterpreamble[u]{\margintoc}
\chapter{Segments} \label{ch:segments}
\langname{} has 26 phonemic segments, comprising 18 consonants and 8 vowels. The segmental inventory is relatively small compared to the cross-linguistic median, but the ratio of consonants to vowel qualities is the same. \marginnote[*2]{Discounting tense vowels, the C/VQ ratio in \langname{} is 3.6; \href{https://wals.info/chapter/3}{Maddieson (2013)} gives the cross-linguistic median as 3.5.}

There are two major dialects of \langname{}: the Port Standard dialect and the Frontier Standard dialect. Port Standard is more conservative, formed from the dialect region around the northern coastal cities, whereas Frontier Standard is more innovative, formed from the newer dialect region in the inland cities. The segmental representation of \langname{} phonology is dialect-neutral, but dialectal realizations of phonemes are discussed when notable.

\section{Consonants}
\langname{} has 18 consonant phonemes. There are four distinguished places of articulation: \emph{labial}, \emph{plain alveolar}, \emph{sibilant alveolar}, and \emph{dorsal}, and four distinguished methods of articulation: \emph{nasal}, \emph{stop}, \emph{fricative}, and \emph{glide}. Stops can be further divided into tenuis and voiced, although the voiced counterparts are marginal.

\begin{table}[h] \centering
    \begin{tabular}{c|cccccccc}
        \toprule
        & \multicolumn{2}{c}{\bf Labial} & \multicolumn{2}{c}{\bf Alveolar} & \multicolumn{2}{c}{\bf Sibilant} & \multicolumn{2}{c}{\bf Dorsal} \\
        \midrule
        \bf{Nasal}     & m & \it\rzc m & n & \it\rzc n & n͡z & \it\rzc z \\
        \mr{\bf Stop}  & p & \it\rzc p & t & \it\rzc t & t͡s & \it\rzc c & k & \it\rzc k \\
                       & b & \it\rzc b & d & \it\rzc d & d͡z & \it\rzc j & g & \it\rzc g \\
        \bf{Fricative} & f & \it\rzc f & ɬ & \it\rzc l & s & \it\rzc s & x & \it\rzc h \\
        \bf{Glide}     & w & \it\rzc v & ɹ & \it\rzc r & & & j & \it\rzc y \\
        \bottomrule
    \end{tabular}
    \caption{Consonants}
    \end{table}

\paragraph{Nasal sibilant}
The nasal sibilant /n͡z/ is a typologically odd segment. The prototypical transcription of this sound is [z̃], a voiced fricative with simultaneous nasal release; however, phonetic research\marginnote{See \href{https://www.icacommission.org/Proceedings/ICA1998Seattle/pdfs/vol_4/2921_1.pdf}{Ohala et al. (1998)} for further discussion of nasal fricatives.} suggests that true fricatives cannot be fully nasalized. Thus, this segment's typical realization may be better represented as a slightly fricated approximant [\tsps{n}z̞̃]. Speakers of Frontier Standard often realize it simply as [z], with inconsistent prenasalization. % or a fully nasalized approximant [\tsps{n}ɹ̃].
Nevertheless it patterns as a nasal sibilant in distribution and morphophonemic processes and is thus phonemically represented as /n͡z/.

\paragraph{Labial fricative}
The fricative /f/ has a marginal distribution, mostly appearing in loan words, names, and onomatopoeia. Most instances of historical /f/ underwent debuccalization and were subsequently lost. Although /x/ also appears frequently in onomatopoeia, it has a more widespread distribution and is not considered marginal.

\paragraph{Sibilant affricate}
Although the phone [t͡s] appears commonly in the corpus, its underlying form is not always the phoneme /t͡s/. \marginnote{Affrication of stop-fricative clusters and schwa deletion are common processes that yield [t͡s], discussed further in §\ref{sub:affrication} and §\ref{sub:schwa_delete}.} The underlying phoneme is usually elucidated by inflection. For example, both \rz{tsekla} “steering wheel” and \rz{cekla} “great uncle” share the same surface form, [ˈt͡sɛkɬə]. However, when inflected for the plural, the former becomes [ˌtasəkˈɬaz̃ɹ] and the latter [t͡səkˈɬaz̃ɹ]. As such \rz{tsekla} is analyzed as /tasɛkɬa/, whereas \rz{cekla} is analyzed as /t͡sɛkɬa/. 

\paragraph{Voiced stops}
Voiced stops are often the result of nasal clusters. \marginnote[*-1]{Voicing is discussed further in §\ref{sub:voicing}.} They are often realized with slight prenasalization, especially in stressed onsets.

\paragraph{Lateral fricative}
Canonically represented by /ɬ/, the lateral fricative is rather diverse in its crossdialectal realizations. Historically, \pz{*l} became fricated when it dissimilated away from \pz{*r}. % which had become [ɹ] due to its own dissimilation from \pz{*z}. 
In Port Standard, /ɬ/ is still realized as [l] except in stressed onsets. % In some frontier dialects, it shifted further to [ʃ].

\section{Vowels}
\langname{} has 8 vowel phonemes, 2 high and 6 low. Low vowels are additionally split into tense and lax vowels.

\begin{table}[h] \centering
    \begin{tabular}{c|cccccc}
        \toprule
        & \multicolumn{2}{c}{\bf Front} & \multicolumn{2}{c}{\bf Mid} & \multicolumn{2}{c}{\bf Back} \\
        \midrule
        \bf High     & i & \it\rzc i &   &           & u & \it\rzc u \\
        \mr{\bf Low} & eː & \it\rzc ę & aː & \it\rzc ą & oː & \it\rzc ǫ \\
                     & ɛ & \it\rzc e & a & \it\rzc a & ɔ & \it\rzc o \\
        \bottomrule
    \end{tabular} 
    \caption{Vowels}
\end{table}

\paragraph{Vowel neutralization} 
Low vowels are reduced to schwa in unstressed syllables.\marginnote{Neutralization is discussed further in §\ref{sub:schwa_neutral}.} Schwa is not phonemic, but neutralization is common, so it appears frequently throughout the corpus. In speech, the underlying vowel becomes evident when stress is shifted due to morphological processes. 

\paragraph{Tense vowels}
Low tense vowels historically derive from nasalized segments, and are still realized with nasalization in the Port Standard dialect. On the other hand, the Frontier Standard dialect typically breaks these vowels, realizing them as falling diphthongs.\marginnote{Many of the nonstandard frontier dialects further merge /aː/ and /oː/.} Because of the disparity of cross-dialect pronunciations, the primary distinguishing feature is tenseness.

Tense vowels are mostly found in open syllables. Lax vowels are usually found in closed syllables.

\section{Stress}
 Stress typically falls on the penultimate syllable of a word. Atypical vowel stress, most common in loan words and compounds, is marked with an acute. \marginnote{For example, \rz{kemu} “fruit slice” is predictable and unmarked, but \rz{meatrę́} “limb” has atypical final stress and is thus marked.}

Stress is lexically productive. Words phonemically distinguished only by stress often have further differences in realization caused by reduction. For example, although \rz{rasrę́} “netting” and \rz{resrą} “owl” have the same segments, /rɛsreː/, the stress pattern obscures that.

Stress is also morphologically productive, distinguishing between unmarked and direct evidence in nouns. Affixation, compounding, and other stress-shifting processes often cause stress-based minimal pairs to become homophones.

Stressed vowels have three phonetic differences from unstressed vowels. First, they typically have a rising pitch. Second, they are typically longer than unstressed vowels. Third, onsets before long vowels have a longer VOT than other onsets, a manifestation of slight aspiration or breathiness.

Secondary stress falls on alternating syllables starting from primary stress and spreading left. For example, \rz{kagęsa} “army” has regular stress on the penultimate syllable, but when inflected in plural form, it surfaces as \rz{kegąsazr} [ˌkɛ.gəːˈsaː.z̃ɹ]. Secondary stress prevents the reduction of /e a o/ to [ə], but, unlike primary stress, does not cause length or VOT increase.

\setchapterpreamble[u]{\margintoc}
\chapter{Processes}
\section{Consonants} 
\subsection{Voicing} \label{sub:voicing}
Across morpheme boundaries, clusters composed of a nasal and stop are realized as voiced stops, occasionally prenasalized. \marginnote{These clusters assimilate in place to the stop, so //np// surfaces as /b/, not /d/.} Clusters where the nasal is the onset of a syllable, not a coda, do not assimilate, so //pn// is still realized /pn/. 

The assimilation process is the source of all native words with voiced plosives, so the phonemes have a limited and predictable distribution. \marginnote{As such, voiced stops are only marginally phonemic.} Foreign words with voiced plosives are often loaned with epenthetic vowels that break up clusters, making the words more closely match native phonotactics. % Indeed, most native words with voiced plosives are transparent compounds, such as \rz{ebar} “below” (← \rz{ez} + \rz{par}) or \rz{kagęsa} “army” (← reduplication of \rz{kęsa}).

Some conjugations of \rz{m}-stem verbs are spelled with a word-final voiced plosive, but the stop is still typically realized as a medial cluster. For example, \rz{seg} “tell themselves” is underlying //sɛm+k// and realized as [sɛ\tsps{(n)}g\tsps{ə}]. \marginnote[*-2]{Some younger speakers always elide the final schwa, especially in the south.}

\subsection{Affrication and assibilation} \label{sub:affrication}
Across morpheme boundaries, clusters of /t/ and a fricative are realized as a sibilant affricate. Clusters of //t+s//, //t+f//, //t+ɬ// and //t+x// all neutralize to /t͡s/, romanized as \rz{c}. 

Clusters of /t/ and the other sibilant phonemes also affricate, but instead of neutralizing to /t͡s/, they have other realizations. Clusters of //t+n͡z// are realized as [t͡s̞̃], romanized as \rz{cz}. \marginnote[*-2]{[t͡s̞̃] is notoriously hard for non-native speakers to pronounce and is often used as a shibboleth.} Clusters of //t+ts// are realized as [t͡sː], romanized as \rz{cc}. In practice, however, these realizations are rare, even when speaking the standardized dialect; most speakers simply render both clusters as [t͡s].

Similar to the affrication process, clusters of sibilants and non-sibilant fricative clusters also neutralize. Unvoiced sibilants //t͡s s// clustering with //s f x ɬ// become /s/,\marginnote{Sporadically realized as a long [sː] or [z̞̃ː].} romanized as \rz{cc} or \rz{ss} depending on the underlying phoneme. The voiced sibilant //n͡z// becomes /n͡z/ instead, romanized as \rz{zz}.

\section{Vowels}
\subsection{Low vowel neutralization} \label{sub:schwa_neutral}
Low vowels are reduced to [ə] in unstressed syllables. For tense vowels, the schwa is metrically longer [əː]. Primary or secondary stress can prevent reduction. The high vowels /i u/ never reduce to schwa.

\subsection{Schwa deletion} \label{sub:schwa_delete}
The reduced vowel [ə] is often deleted between consonants,\marginnote{Tense schwa [əː] rarely undergoes deletion as it is typically longer than [ə].} especially between plosives and sibilants. For example, \rz{ksarat} /kɔsaɹat/ is usually [ksaɹət], and spelled accordingly. When clustering in this way, both //t͡s// and //s// become /s/. The elision process results in word-final or word-initial [s] or [z̃] being the only common syllable-internal clusters. However, these clusters are not consistently realized, and occasionally have an epenthetic schwa, especially word-finally. 

\setchapterpreamble[u]{\margintoc}
\chapter{Phonotactics}
\section{Roots}
Modern native lemma derive from CV or CV(C)CV roots. Some roots have null initials, but they are comparatively rare. Roots allow CC clusters that aren't allowed across morpheme boundaries.

Roots are nominal by default, and in the corpus many roots appear in bare form as nouns. Stems reflect reduced derivational morphemes that transformed roots into verbs or other nouns.\marginnote{Stem endings can often be reconstructed with vague meanings.}

The most prominent verb stem is -\rz{t}, forming both transitive and intransitive verbs with a variety of meanings. Another common stem is -\rz{m}, which also derives stems of either valency class and has a vaguely frequentative connotation. The other rarer verb stems are the intransitive -\rz{k} and the transitive -\rz{l} and -\rz{y}, which encoded some kind of telicity distinction.

Outside of bare roots, -\rz{s} stem is the most common for nouns. There are also -\rz{c} stems and -\rz{z} stems, which likely derived collective and agentive nouns, respectively. All nouns belonging to the sibilant stems are neuter gender. The non-sibilant -\rz{r} stem can form nouns of either gender.

\paragraph{Function words}
A smaller set of phonemic segments are allowed in functional morphemes. Neither the labials /p f m w/ nor the high vowel /u/ appear in affixes, particles, or prepositions.\marginnote{Because /u/ and /w/ pattern similarly, some phonemic analyses conflate them.} In the fossilized partial reduplication process, the reflexes of labial consonants are either /k/ (← /p f/) or /n/ (← /m w/). The high vowel /u/ lowers to /o/, often simply realized as [ə].