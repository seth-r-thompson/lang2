\setchapterpreamble[u]{\margintoc}
\chapter{Segments}
\section{Consonants}
\langname{} has 18 consonant phonemes. There are four distinguished places of articulation: \emph{labial}, \emph{plain alveolar}, \emph{sibilant alveolar}, and \emph{dorsal}, and four distinguished methods of articulation: \emph{stop}, \emph{fricative}, \emph{nasal}, and \emph{approximant}. Stops can be further divided into tenuis and voiced, although the voiced counterparts are marginal.

\begin{table}[h] \centering
    \begin{tabular}{c|cccccccc}
        \toprule
        & \multicolumn{2}{c}{\bf Labial} & \multicolumn{2}{c}{\bf Alveolar} & \multicolumn{2}{c}{\bf Sibilant} & \multicolumn{2}{c}{\bf Dorsal} \\
        \midrule
        \bf{Stop}           & p & \it\rzc p & t & \it\rzc t & t͡s & \it\rzc c & k & \it\rzc k \\
                            & b & \it\rzc b & d & \it\rzc d & d͡z & \it\rzc j & g & \it\rzc g \\
        \bf{Fricative}      & f & \it\rzc f & ɬ & \it\rzc l & s & \it\rzc s & x & \it\rzc h \\
        \bf{Nasal}          & m & \it\rzc m & n & \it\rzc n & n͡z & \it\rzc z \\
        \bf{Approximant}    & w & \it\rzc v & ɹ & \it\rzc r & & & j & \it\rzc y \\
        \bottomrule
    \end{tabular}
    \caption{Consonants}
    \end{table}

\paragraph{Nasal Sibilant}
The nasal sibilant /n͡z/ is a typologically odd segment. The prototypical representation of this sound is [z̃], a voiced fricative with simultaneous nasal release; however, phonetic research\marginnote{See \href{https://www.icacommission.org/Proceedings/ICA1998Seattle/pdfs/vol_4/2921_1.pdf}{Ohala et al. (1998)} for further discussion of nasal fricatives.} suggests that true fricatives cannot be fully nasalized. Thus, this segment's typical realization may be better represented as a slightly fricated approximant [\tss{n}z̞̃], although some speakers may realize it as a weakly nasalized [\tss{n}z] or a fully nasalized approximant [\tss{n}ɹ̃]. Nevertheless it patterns as a nasal sibilant in distribution and morphophonemic processes and is thus phonemically represented as /n͡z/.

\paragraph{Labial Fricative}
The fricative /f/ has a marginal distribution, mostly appearing in loan words, names, and onomatopoeia. Most instances of historical /f/ underwent debuccalization and were subsequently lost. Although /x/ also appears frequently in onomatopoeia, it has a more widespread distribution and is not considered marginal.

\begin{kaobox}[frametitle=\sc todo:]
    This next section about the sibilant affricate could use some wording cleanup. It should also cover all the instances of /t͡s/ that are [s].
\end{kaobox}

\paragraph{Sibilant Affricate}
Although the phone [t͡s] is rather common in the corpus, its underlying form is not always the phoneme /t͡s/. \marginnote{Affrication of stop-fricative clusters and pre-[s] schwa deletion are common processes that yield [t͡s], discussed further in §\ref{sec:conso_morphono}.} The underlying phoneme is usually elucidated by inflection. For example, both \rz{tsekla} “steering wheel” and \rz{cekla} “great uncle” share the same surface form, [ˈt͡sekɬə]. However, when inflected for the plural, the former becomes [ˌtasəkˈɬaz̃əɹ] and the latter [t͡səkˈɬaz̃əɹ]. As such \rz{tsekla} is analyzed as /tasekɬa/, whereas \rz{cekla} is analyzed as /t͡sekɬa/. This analysis parallels liaison in some languages.

% From a contrast perspective, the three most important features for segments are labial, sibilant,\marginnote{The sibilant place of articulation is a sprachbund feature found in other Peninsular languages including East Cape.} and nasal. Labial sounds form a restricted class that does not appear in non-content morphemes. Sibilants and nasals condition morphophonemic changes across morpheme boundaries, and sibilants are an important phonotactic class, as well.

\section{Vowels}
\langname{} has 8 vowel phonemes, 5 plain and 3 nasalized. 

\begin{table}[h] \centering
\begin{tabular}{c|cccccccccccc}
    \toprule
    & \multicolumn{4}{c}{\bf Front} & \multicolumn{4}{c}{\bf Mid} & \multicolumn{4}{c}{\bf Back} \\
    \midrule
    \bf High & i & \it\rzc i & & & & & & & u & \it\rzc u \\
    \bf Mid & e & \it\rzc e & ẽ & \it\rzc ę & (ə) & \it\rzc a & (ə̃) & \it\rzc ą & o & \it\rzc o & õ & \it\rzc ǫ \\
    \bf Low & & & & & a & \it\rzc a & ã & \it\rzc ą \\
    \bottomrule
\end{tabular} 
\caption{Vowels}
\end{table}

\begin{kaobox}[frametitle=\sc todo:]
This section about vowel neutralization probably belongs in morphophonology? It also has to do with the supersegment stress.
\end{kaobox}

\paragraph{Vowel Neutralization} 
Mid and low vowels /e a o/ and their nasal counterparts are reduced to [ə ə̃] in unstressed syllables. Schwa is not phonemic, but neutralization is common, so it appears frequently throughout the corpus. In speech, the underlying vowel becomes evident when stress is shifted due to morphological processes.

\section{Allophony}

\setchapterpreamble[u]{\margintoc}
\chapter{Supersegments}
\section{Stress}
Stress in \langname{} is lexically and morphologically productive. Stress typically falls on the penultimate syllable of a word. Atypical vowel stress, most common in loan words, is marked with an acute. \marginnote{For example, \rz{tąka} “tree” has typical stress and is unmarked, but \rz{tąká} “reindeer” has atypical ultimate stress and is thus marked.} Stress is also morphologically productive, distinguishing between unmarked and distal deixis in nouns. Affixation, compounding, and other stress-shifting processes often cause stress-based minimal pairs to become homophones.

Stressed vowels have three phonetic differences from unstressed vowels. First, they typically have a rising pitch. Second, they are typically longer than unstressed vowels. Third, onsets before long vowels have a longer VOT than other onsets, a manifestation of slight aspiration or breathiness.

Secondary stress falls on alternating syllables starting from primary stress and spreading left. For example, \rz{kagęsa} “army” has regular stress on the penultimate syllable, but when inflected in plural form, it surfaces as \rz{kegąsazar} [ˌke.gə̃ˈsa.z̃əɹ]. Secondary stress prevents the reduction of /e a o/ to [ə], but, unlike primary stress, does not cause length or VOT increase.

\setchapterpreamble[u]{\margintoc}
\chapter{Morphophonology}
\begin{kaobox}[frametitle=\sc todo:]
This all just got recently moved from allophony because it's better analyzed as a morphophonological phenomenon, probably. So some of it will have to be re-written with that framing in mind. Probably means a lot more double slashes and “it only occurs across morpheme boundaries” type stuff.
\end{kaobox}
\section{Consonants} \label{sec:conso_morphono}
\paragraph{Voicing}
Nasal and stop clusters are realized as voiced stops, occasionally prenasalized. The resulting phones [\tss{(m)}b \tss{(n)}d \tss{(n)}d͡z \tss{(ŋ)}g] are romanized \rz{b d j g}. \marginnote{These clusters assimilate in place to the stop, so /np/ surfaces as [b], not as [d].} Clusters where the nasal is the onset of a syllable, not a coda, do not assimilate, so /pn/ is still realized [pn]. 

Although the assimilation process is most common across morpheme boundaries or in loan words, voiced plosives can be found in some native morphemes. However, since these phones only occur word-medially in limited and predictable distribution, they are not traditionally considered phonemic. \marginnote{Some scholars argue that voiced plosives are phonemic or becoming phonemic because of their presence in loan words and verb conjugations.} Most native words with voiced plosives are transparent compounds, such as \rz{ebar} “below” (← \rz{ez} + \rz{par}) or \rz{kagęsa} “army” (← reduplication of \rz{kęsa}).

Some words, especially some conjugations of \rz{m}-stem verbs, are spelled with a word-final voiced stop, but the stop is still typically realized as a medial cluster. For example, \rz{sed} “they tell me” is underlying /semt/ and realized [se\tss{(n)}d(ə)]. For some speakers, especially younger speakers or those in informal contexts, the final schwa is elided.

\begin{kaobox}[frametitle=\sc todo:]
Affrication and assibiliation would make more sense if it was only across morpheme boundaries, but I have some words that I like where it's morpheme-internal. How should I handle that? It could be reworked as a diachronic process, made phonemic, a morphophonemic process---but it needs more thought.
\end{kaobox}

\paragraph{Affrication}
Alveolar plosive and fricative clusters are realized as a sibilant affricate. Clusters of /ts/, /tf/, /tɬ/ and /tx/ all neutralize to [t͡s], romanized as \rz{c}. The /tn͡z/ cluster likewise affricates, but is realized as [t͡s̞̃], romanized as \rz{cz}. \marginnote[*-2]{The voiceless nasal affricate is notoriously hard for non-native speakers to pronounce and is often used as a shibboleth.}

\paragraph{Assibilation}
Sibilant and non-sibilant fricative clusters are realized as sibilants. Unvoiced sibilants /t͡s s/ clustering with /f x ɬ/ are realized as [sː], romanized as \rz{cc} or \rz{ss} depending on the underlying phoneme. The voiced sibilant /n͡z/ instead is realized as [z̞̃ː] in such clusters, romanized as \rz{zz}.

\section{Vowels}
\paragraph{Schwa Deletion}
The reduced vowel [ə] is often deleted between consonants,\marginnote{Nasalized schwa [ə̃] rarely undergoes deletion as it is typically longer than [ə].} especially non-alveolar stops, and the sibilants /t͡s s n͡z/. For example, \rz{ksarat} /kosaɹat/ is commonly realized as [ksaɹət], and spelled accordingly. The elision process results in word-final or word-initial [s]\marginnote{Note that /t͡s/ surfaces [s] in these environments.} or [z̃] being the only syllable-internal clusters. However, these clusters are not consistently realized, and occasionally have an epenthetic schwa, especially word-finally. 

\begin{kaobox}[frametitle=\sc todo:] 
Something here about word-final clusters. I haven't been consistent about when it's spelled as a cluster or when it's spelled with the vowel, and I need to either commit to the inconsistency, or figure out how that works.
\end{kaobox}

\begin{kaobox}[frametitle=\sc todo:] 
There might end up being an isogloss map of schwa deletion: most dialects delete ahead of sibilants, some ahead of sibilants and approximants, and others in all environments?
\end{kaobox}

\setchapterpreamble[u]{\margintoc}
\chapter{Phonotactics}
\section{Roots}
Modern native lemma derive from CV or CV(C)CV roots. Some roots have null initials, but they are comparatively rare. Roots allow CC clusters that aren't allowed across morpheme boundaries.

Roots are nominal by default, and in the corpus many roots appear in bare form as nouns. Stems reflect reduced derivational morphemes that transformed roots into verbs or other nouns.\marginnote{Stem endings can often be reconstructed with vague meanings.}

The most prominent verb stem is -\rz{t}, forming both transitive and intransitive verbs with a variety of meanings. Another common stem is -\rz{m}, which also derives stems of either valency class and has a vaguely frequentative connotation. The other rarer verb stems are the intransitive -\rz{k} and the transitive -\rz{l} and -\rz{y}, which encoded some kind of telicity distinction.

Outside of bare roots, -\rz{s} stem is the most common for nouns. There are also -\rz{c} stems and -\rz{z} stems, which likely derived collective and agentive nouns, respectively. All nouns belonging to the sibilant stems are neuter gender. The non-sibilant -\rz{r} stem can form nouns of either gender.

\paragraph{Affixes}
A smaller set of phonemic segments are allowed in affixes. Neither the labials /p f m w/ nor the high vowel /u/ appear in true affixes.\marginnote{Because /u/ and /w/ pattern similarly, some phonemic analyses conflate them.} In the fossilized partial reduplication process, the reflexes of labial are either /k/ (← /p f/) or /n/ (← /m w/). The high vowel /u/ lowers to /o/, often simply realized as [ə].