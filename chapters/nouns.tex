\setchapterpreamble[u]{\margintoc}
\chapter{Nouns}
The \langname{} noun phrase is largely analytic, but nouns do inflect for evidence and number.

\section{Evidence} \label{sec:evidence}
Nominal evidence marks the kind of knowledge the speaker has about the referent.\marginnote[*3]{I had this idea, then found out that, as usual, a natural language had it first. Read \href{http://lingpapers.sites.olt.ubc.ca/files/2020/07/11_ICSNL55_Huijsmans_Reisinger_Matthewson_final.pdf}{Huijsmans, Reisinger, and Matthewson (2020)} for more about the Salishan languages.} Verbs and adjectives exhibit agreement for this evidence. There are three evidential categories: \emph{generic}, \emph{direct}, and \emph{indirect}.

Although diachronically related to distal demonstratives, evidential forms are primarily indicators of non-propositional evidentiality, \tit{i.e.} the speaker's evidence of a nominal referent. Evidence in this sense is either \emph{direct} or \emph{indirect}. Direct evidence is a firsthand account, which is typically by sight but can be some other sense. Indirect evidence is largely hearsay or inference. The non-propositional evidentiality system in \langname{} is based on the speaker's knowledge at or prior to speech time,\marginnote{\href{https://ditibhadra.com/chapter_draft.pdf}{Bhadra (2020)} terms this a Type I system.} not the event time.

\subsection{Direct}
Direct evidence signals that the speaker has personal, firsthand experience with the referent. Usually, it means the speaker has met the person, or sensed the object, being referred to. It encodes many senses, including sight, hearing, and touch.

\begin{example}
    \script Nassoin miyin kąsteci Katva z-dahę́s t-hakra.
    \bits nassoi-n miyi-n kęstac-i {Katva ez=dahę́s} tę=hakra
    \gloss king-KN brave-KN leads-KN {Katva peoples} into=battle
    \tr The heroic king (I've met) led our people into battle.
\end{example}

Direct evidence is marked with the suffix \rz{-n}. This suffix attaches before derivational suffixes like -\rz{s} or -\rz{r}.

\subsection{Indirect}
Indirect evidence signals that the speaker has heard of the referent, or can make an educated guess about it. Inference is fairly broad; in practice, it's used for most things the speaker hasn't sensed. 

\begin{examples*}
    \ex \label{ex:indirect_evidence}
        \script egi Matkó aczé samés t-het.
        \bits egi matkó aczé sem-és tę=het
        \gloss just basket:UN two:UN will-3.UN to={be at}
        \tr There'll be just two baskets (I've heard).
        \smoyd https://www.reddit.com/r/conlangs/comments/ic7on8/1314th_just_used_5_minutes_of_your_day/ & 1314
\end{examples*}

Indirect evidence is marked by shifting stress to the final syllable of the word.

\subsection{Generic}
The generic form of the noun is morphologically least marked and used as the citation form. However, in practice its meaning is more semantically restricted than either direct or indirect forms.

The generic form is most often used for gnomic statements.

\begin{example}
    \script Kęsa esyi oc.
    \bits kęsa esyi oc
    \gloss hero good is
    \tr Heroes are good.
\end{example}

The generic is also use for non-specific referents. In (\ref{ex:non-specific}), the use of \rz{vącizr} conveys that there's not any particular group being referenced.

\begin{example} \label{ex:non-specific}
    \script Nassoin semsi vącizr t-kęstat.
    \bits nassoin semsi vącizr tę=kęstat
    \gloss {the king} may mercenary\tsc{:pl} to=hire
    \tr The king may hire mercenaries.
\end{example}

If direct or indirect evidence exists, it's infelicitous to use the generic form.

\section{Number}
Nouns inflect morphologically for an additive plural, but there is also a periphrastic construction used to form associative plurals, and other constructions to convey semantic plurality.

The unmarked form of a noun encodes expected number \marginnote{For example, \rz{parsa} “eyes” defaults to a pair and needs some specifier to be singular.} and thus must be modified by a numeral or appositive to convey some other sense, such as a singular or dual. 

\begin{figure}[h]
    \centering
    \begin{subfigure}{0.4\textwidth}
        \centering
        \includegraphics[width=\textwidth]{same_vases.jpg}
        \caption{\rz{matkozr}}
    \end{subfigure}
    \begin{subfigure}{0.4\textwidth}
        \centering
        \includegraphics[width=\textwidth]{different_vases.jpg}
        \caption{\rz{ezzu metka}}
    \end{subfigure}
    \caption{Additive vs. associative plurals}
\end{figure}

The primary difference in the two plurals is the composition of the set: additive plurals refer to largely homogenous referents, whereas associative plurals refer to largely heterogenous referents.

\subsection{Additive plurals} \label{subsec:additive_plural}
Additive plurals are used for a set of homogenous referents and never heterogenous referents; \eg \rz{matkozr} is “a set of the same (or similar) vase” and never “a set of diverse vases.”\marginnote[*-1]{The second meaning would use the associative plural.}

Additive plurality is indicated through the suffix \rz{-zr}. Morphonologically, this suffix can be thought to precede evidence marking; it circumfixes the direct evidence marker, and bears the final stress of the indirect evidence marker.

\begin{subexamples}
    \baarucols3
    \ex
        \script matkozr
        \gloss basket-PL
    \ex 
        \script matkoznr
        \gloss basket-PL.KN
    \ex
        \script metkazŕ
        \gloss basket-PL.UN
\end{subexamples}

However, marking is optional: a noun can be inferred additively plural from context. \marginnote{Mandatory plural marking is stylistically preferred in formal contexts.} It's especially common to not mark plurals for small, discreetly countable sets or when a referent has been established plural in prior conversation. 

Because of this the plural suffix can be considered partly derivational, and there are other alternative strategies to form plurals.

\subsection{Periphrastic plurality}
Numerals and modifiers are the primary alternate plural strategies. They don't require plural marking on the noun. If the plural is used, it typically conveys a set.

\begin{subexamples}
    \baarucols2
    \ex \script cǫhi ocza
    \bits cǫhi ocza
    \gloss dress two
    \tr two dresses
    \ex \script cąhizr ocza
    \bits cąhi-zr ocza
    \gloss dress-PL two
    \tr two sets of dresses
\end{subexamples}

The noun \rz{tevi} “many” is the most common appositive modifier to use for the plural. It can also be used to mean “each,” especially when used with pronouns.%\marginnote[*-1]{See §...}

\begin{subexamples}
    \baarucols2
    \ex \label{ex:tevi_plural}
        \script vęci tevi
        \tr some mercenaries
    \ex
        \script rat tevi
        \tr each of us
\end{subexamples}

Some nouns have collocative modifiers to specify their plurals, which are noted in the dictionary.

\subsection{Associative plurals}
The associative plural is used for a set of heterogenous referents. For animate (especially human) referents, the meaning is canonically “a person and their associates,” as in (\ref{ex:ezzu_basic}). The focal referent\marginnote[*-1]{The focal referent is the main member of the group; see \href{https://amor.cms.hu-berlin.de/~h2816i3x/Lehre/2007_VL_Typologie/03_Daniel_AssociativePlural.pdf}{Daniel and Moravcsik (2007)}.} (\ie most important) is the marked noun.

\begin{example} \label{ex:ezzu_basic}
    \script Sayanaltal oci ezzu Kanyi z-lanr.
    \bits sayenaltal oci ezzu Kanyi ez=lanr
    \gloss {being ignorant} be ASSC \tsc{name} in=this
    \tr For this, Kanyi and his friends won't be much help.
\end{example}

However, the associative can also have a number of idiomatic, context-specific meanings, usually referring to diverse groups.

\begin{example} \label{ex:ezzu_idiom}
    \script ezzu Mazziznr segi s-yiat saska hakra.
    \bits ezzu mazziznr segi su=yiat {saska hakra}
    \gloss ASSC ballers want:\tsc{refl} to=steal comeback
    \tr The entire league wants to pull off the upset.
\end{example}

The associative plural is marked with the preposition \rz{ezzu}. Like other \rz{su}-derived prepositions, it can head a constituent phrase.\marginnote[*-1]{See §\ref{sec:free_preps} for more on free prepositions.}

\section{Gender}
Although nouns traditionally distinguish \emph{common} and \emph{neuter} gender, this has largely become a prescriptive convention. \marginnote{The gender distinction is more common in literature or academia.} Most neuter words are either loan words or end in -\rz{s}.

Some words only distinguish gender for certain uses or contexts, so dictionaries typically denote if a given usage is expected to require neuter gender.

\section{Adjectives and Apposition}
Adjectives and appositive nouns used as modifiers have similar morphological and syntactic distributions. Adjectives in \langname{} are essentially a subset of nouns that cannot head a noun phrase.

\subsection{Adjectives}
Adjectives agree with their head noun for evidence (but not number) and syntactically appear before other modifiers like appositives or prepositionals.

\begin{example}
    \script tseklan sovan
    \bits tsekla-n sova-n
    \gloss wheel-KN new-KN
    \tr new vehicle
\end{example}

Adjectives form a small, closed class; there are approximately 20 adjectives. Most adjectives are color words or broadly describe physical states.

\subsection{Apposition}
Apposition is common due to the small adjective class present in \langname{}. Like adjectives, appositive nouns show agreement with the noun they modify.

\begin{example}
    \script tseklan ahkan
    \bits tsekla-n akha-n
    \gloss wheel-KN wise-KN
    \tr trusty vehicle
\end{example}

Apposition is used in a number of collocative constructions, but is distinct from compounding largely because of stress patterns and morphophonological processes.\marginnote[*-2]{Compounds shift stress and show cross-morpheme sound changes.}

\subsection{Inalienable possession}
Apposition is also used for inalienable possession. 

\begin{subexamples}
    \baarucols2
    \ex 
        \script hora kęsa
        \tr soldier's wrist
    \ex \label{ex:alien_poss}
        \script hora im-kęsa
        \tr soldier's skill
\end{subexamples}

Inalienable possession is typical for body parts and relatives. As in (\ref{ex:alien_poss}), more metaphorical senses of these nouns have regular possession marking.

\section{Pronouns}
Like adjectives, pronouns are morphologically and syntactically similar to other nouns.

\begin{margintable} \centering
    \begin{tabular}{c|c}
        \toprule
        \tsc{1st} & \rz{rat} \\
        \tsc{2nd} & \rz{a} \\
        \tsc{cmn} & \rz{sec} \\
        \tsc{ntr} & \rz{moc} \\
        \bottomrule
    \end{tabular} 
    \caption{Personal pronouns}
\end{margintable}

\langname{} has four personal pronouns. Personal pronouns are usually anaphoric, referring to some earlier referent in discourse. Although verbs agree with person, they are rarely omitted except in rapid speech like commands.

Personal pronouns can inflect for plurality, but it's not common; typically the inflection is reserved for clarity or emphasis. They almost never inflect for evidence, except to agree with head nouns in appositive constructions.

Unlike nouns, personal pronouns can be used appositively for both alienable and inalienable possession. Constructions with \rz{im} are more emphatic.

\begin{subexamples}
    \baarucols2
    \ex \script Iamaj ratn pici.
    \bits iamaj rat-n pici
    \gloss net:KN 1:KN be:KN
    \tr It's my net.
    \ex \script Iamaj m-rat pici.
    \bits iamaj im=rat pici
    \gloss net:KN of=1 be:KN
    \tr The net is \tit{mine}.
\end{subexamples}

Furthermore, clauses can be intensified with a reduplicated \rz{im} construction.

\begin{example}
    \script rat m-rat satr lanr.
    \bits rat im=rat satr lanr
    \gloss 1 of=1 do this
    \tr I myself did this.
\end{example}

\begin{margintable} \centering
    \begin{tabular}{c|c}
        \toprule
        \tsc{ntr} & \rz{lar} \\
        \tsc{cmn} & \rz{mans} \\
        \bottomrule
    \end{tabular} 
    \caption{Impersonal pronouns}
\end{margintable}

There are also two impersonal pronouns. Their meanings are similar to “something” and “somebody,” respectively. The most common use for them is as dummy arguments of verbs.

\section{Phrasal syntax}
Noun phrases are predominantly head-initial. Generally speaking, syntactically simpler constituents occur before more syntactically complex ones.

\begin{example}
    head → adjective → number → appositive → prepositional
\end{example}