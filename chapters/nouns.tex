\setchapterpreamble[u]{\margintoc}
\chapter{Nouns}
The \langname{} noun phrase is largely analytic, but nouns do inflect for deixis and number. % Nouns have three broad inflection patterns, largely related to the way they inflect for plurality.

% \section{Stems}
% Nouns are broadly divided into three stems based on their inflectional patterns: \emph{vocalic}, \emph{\rz{s}-stems}, and \emph{\rz{r}-stems}. Rarely noun stems will end in other consonants, but these have no discernible shared patterns and often comprise newer loans.

% Most noun stems are vocalic stems; typically these end in mid vowels, although some do end in a high vowel. Vocalic stems are usually common gender, excluding loanwords, which are prescriptively assigned neuter gender even if they end in a vowel. 

% Neuter nouns, on the other hand, are often either \rz{r}-stems or \rz{s}-stems, the latter being more common. \marginnote{\rz{S}-stems can end in any sibilant, typically \rz{-s} but also \rz{-c} or \rz{-z}.} Although no longer morphologically productive, the endings on \rz{s}-stems and \rz{r}-stems derive from historical derivation processes. Many roots are reflected in both endings, but the shared meaning between them is not always transparent.

% \subsection{Vocalic stems}
% Vocalic stems have fairly regular, agglunative inflection patterns. However, the proximal plural is shortened to \rz{-rran} for most speakers.

% \begin{table}[h] \centering
%     \begin{tabular}{c|ccc}
%         \toprule
%         & \bf Generic & \bf Proximal & \bf Distal \\
%         \midrule
%         \bf \sc sg & \it\rzc metka & \it\rzc metkan & \it\rzc matkó \\
%         \bf \sc pl & \it\rzc matkozar & \it\rzc matkorran & \it\rzc matkazár \\
%         \bottomrule
%     \end{tabular}
%     \caption{Inflection of vowel-stem \rz{metka} “bowl”}
%     \label{tab:metka_inflection}
% \end{table}

% All vocalic forms share the same endings, but many will have an unpredictable final vowel due to stress-based mid vowel neutralization. There is no way to predict the final vowel from the uninflected lemma, so learners often resort to memorization.

% \subsection{Neuter stems}
% Neuter stems share a common inflection pattern. For both neuter stems, the proximal surfaces as /on/ instead of /n/. The proximal plural also shortens for neuter stems, but takes the form \rz{-zza-}, influenced by the assibilation morphophonological process. The shared plural form for both stems can create homophones.\marginnote[*-2]{Some strategies to resolve homophones are covered in §\ref{subsec:additive_plural}.} 

% \begin{table}[h] \centering
%     \begin{tabular}{c|ccc}
%         \toprule
%         & \bf Generic & \bf Proximal & \bf Distal \\
%         \midrule
%         \bf \sc sg & \it\rzc retus & \it\rzc ratusan & \it\rzc ratús \\
%         \bf \sc pl & \it\rzc ratuzzar & \it\rzc ratuzzan & \it\rzc ratuzzár \\
%         \bottomrule
%     \end{tabular}
%     \caption{Inflection of \rz{s}-stem \rz{retus} “blade”}
% \end{table}

% \begin{table}[h] \centering
%     \begin{tabular}{c|ccc}
%         \toprule
%         & \bf Generic & \bf Proximal & \bf Distal \\
%         \midrule
%         \bf \sc sg & \it\rzc pebar & \it\rzc pabaran & \it\rzc pabár \\
%         \bf \sc pl & \it\rzc pabazzar & \it\rzc pabazzan & \it\rzc pabazzár \\
%         \bottomrule
%     \end{tabular}
%     \caption{Inflection of \rz{r}-stem \rz{pebar} “garden”}
% \end{table}

\section{Deixis}
Nominal deixis has a variety of uses, including evidentiality, distance, familiarity, and topicality.\marginnote{I had this idea, then found out that, as usual, a natural language had it first. Read \href{http://lingpapers.sites.olt.ubc.ca/files/2020/07/11_ICSNL55_Huijsmans_Reisinger_Matthewson_final.pdf}{Huijsmans, Reisinger, and Matthewson (2020)} for more about the Salishan languages.} Verbs and adjectives exhibit agreement for deictic reference. There are three deictic categories: \emph{generic}, \emph{proximal}, and \emph{distal}.

Although diachronically related to demonstratives, spacial deixis is only a secondary use of the proximal and distal forms. Instead, they are primarily indicators of non-propositional evidentiality, \tit{i.e.} the speaker's evidence of a nominal referent. Evidence in this sense is visual or non-visual, the latter largely encoding hearsay or inference. The non-propositional evidentiality system in \langname{} is semantically based on the speaker's knowledge at or prior to speech time.\marginnote{\href{https://ditibhadra.com/chapter_draft.pdf}{Bhadra (2020)} terms this a Type I system.} 

\subsection{Generic}
The generic form of the noun is morphologically least marked and used as the citation form.

The generic form is most often used for gnomic statements.

\begin{example}
    \script Kęsa esyi oc.
    \bits kęsa esyi oc
    \gloss hero is good
    \tr Heroes are good.
\end{example}

% The referent \rz{kęsa} is unmarked to convey that the speaker means heroes in the general sense, not a specific person. Compare (\lastx) with (\nextx), which both refer to a specific hero.

% \begin{gloss}
%     \a \begingl
%         \glpreamble Kęsan esyin ací. \endpreamble
%             kęsa-n[hero-\tsc{prox}]
%             esyi-n[good-\tsc{prox}]
%             ot-s-í[be-\tsc{3c-prox}]
%         \glft “The hero (I know) is good.”
%     \endgl
%     \a \begingl
%             \glpreamble Kąsá asyí aséc. \endpreamble
%                 kęsa[hero\tbs\tsc{dist}]
%                 esyi[good\tbs\tsc{dist}]
%                 ot-s[be-\tsc{3c}\tbs\tsc{dist}]
%             \glft “The hero (I've heard of) is good.”
%     \endgl
% \end{gloss}

If direct or indirect evidence exists, it's infelicitous to use the generic form.

\subsection{Proximal}
The proximal form of a noun is used when the speaker is certain, nearby, or familiar with the noun.
%It can also be used for the conversational topic.
This form most commonly denotes direct evidence, meaning the speaker has personal experience with the marked noun. It is marked with the suffix \rz{-n}. \marginnote[*-2]{\rz{-n} is morphophonemically ⫽on⫽, where ⫽o⫽ doesn't surface for vocalic stems.}

\paragraph{Direct evidence}
The canonical meaning of the proximal form is direct evidence, often translated as “I saw.” 

\begin{kaobox}[frametitle=\sc todo:]
    The definiteness constructions probably need to be reworked to square better with (a) the stuff I've learned about definiteness and (b) the use of the deictic forms for topic/focus.
\end{kaobox}

% \paragraph{Definiteness}
% Proximal forms can be used to describe the definiteness of a referent. This construction is only used for weak, uniqueness-based definiteness (eg. “the Moon”), never for strong, anaphoric definiteness (eg. “the book”). For strong definiteness, the noun \rz{sin} “???” is used alongside distal form, as in (next example b).

% \begin{examples*}
%     \ex 
%         \script Nassoin kęstací su-kagęsa su-kagę́stapa.
%         \bits nassoi-n kęstací su=kagęsa su=kagę́stapa
%         \gloss king-\tsc{prox} commands and=army and=navy
%         \tr The king (we know) commands both army and navy.”
%     \ex 
%         \script sah ez-Rosam pít ató sin.
%         \bits sah ez=rosam pít ató sin
%         \gloss about in=cook be grain ???
%         \tr The rice (you mentioned) is about to be cooked.
%         \smoyd https://www.reddit.com/r/conlangs/comments/kck1hi/1381st_just_used_5_minutes_of_your_day/ & 1381
% \end{examples*}

\subsection{Distal}
The distal form of a noun is used when the speaker is uncertain, far, or unfamiliar with the noun. 
%It can also be used for the conversational focus.
This form typically denotes indirect evidence, including inference, meaning the speaker has heard of or can make an educated guess about the existence of the marked noun. Reported deixis is marked by shifting stress to the ultimate syllable of the word.

\paragraph{Indirect Evidence}
The prototypical meaning of the distal form is indirect evidence, often translated as “heard about” or “they said.” As in (\ref{ex:indirect_evidence}), this evidence is encoded into the clause via the subject and the predicate that agrees with it.

\begin{examples*}
    \ex \label{ex:indirect_evidence}
        \script egi Matkó aczé samés lar t-het.
        \bits egi matkó aczé sem-és lar tę=het
        \gloss just basket\tbs\tsc{dist} two\tbs\tsc{dist} will-\tsc{3.dist} there to={be at}
        \tr She said there'll be just two baskets.
        \smoyd https://www.reddit.com/r/conlangs/comments/ic7on8/1314th_just_used_5_minutes_of_your_day/ & 1314
\end{examples*}

\section{Number}
Nouns inflect morphologically for an additive plural, but there is also a syntactic construction used to form associative plurals. The unmarked form of a noun encodes expected number, \eg \rz{parsa} “eyes” which defaults to a pair and must be modified by a numeral or appositive to specify a singulative. 

\begin{figure}[h]
    \centering
    \begin{subfigure}{0.4\textwidth}
        \centering
        \includegraphics[width=\textwidth]{same_vases.jpg}
        \caption{\rz{matkozar}}
    \end{subfigure}
    \begin{subfigure}{0.4\textwidth}
        \centering
        \includegraphics[width=\textwidth]{different_vases.jpg}
        \caption{\rz{ezzu metka}}
    \end{subfigure}
    \caption{Additive vs. associative plurals}
\end{figure}

The primary difference in the two plurals is the composition of the set: additive plurals refer to largely homogenous referents, whereas associative plurals refer to largely heterogenous referents.

\subsection{Additive plurals} \label{subsec:additive_plural}
Additive plurals are used for a set of homogenous referents and never heterogenous referents; \eg \rz{matkozar} is “a set of the same (or similar) bowl” and never “a set of diverse bowls.”\marginnote{The second meaning would use the associative plural.}

Additive plurality is indicated through the suffix \rz{-zar}. Morphological marking is optional and a noun can be inferred additively plural from context. \marginnote{Mandatory plural marking is stylistically preferred in formal contexts.} As such marking is less common for small, discrete or easily countable sets or when a referent has been established plural in prior conversation. However, speakers are not always consistent with marking.

Morphonologically, plural marking can be thought to precede deictic marking; plural distal nouns have accent placed on the plural suffix. As seen in Table \ref{tab:metka_inflection}, \rz{matkozar} becomes \rz{matkazár}, not \rz{*matkózar}.

\subsection{Other plural strategies}
Because the suffixes of \rz{s}-stems and \rz{r}-stems merge in the plural, some minimal pairs are rendered homophones when inflected. There are a few strategies to combat homophony. 

\paragraph{Adjectives}
Speakers sometimes employ \rz{tevi} “many” as an appositive modifier for the singular form, as in (\ref{ex:tevi_plural})

\paragraph{Explicit number}
When number is specified with a numeral, the noun is not marked for plurality, as in (\ref{ex:number_plural}).

\begin{subexamples}
    \baarucols2
    \ex \label{ex:tevi_plural}
        \script vęci tevi
        \tr some mercenaries
    \ex \label{ex:number_plural}
        \script vęci ocza
        \tr two mercenaries
\end{subexamples}

\subsection{Associative plurals}
Unlike the additive plural, the associative plural is not marked morphologically; it is nonetheless rather prevalent. The preposition \rz{ezzu} conveys the associative meaning.\marginnote{Like \rz{su}, \rz{ezzu} can appear stranded. See §\ref{subsec:u_and_su}.}

The associative is used for a set of heterogenous referents. For animate (especially human) referents, the meaning is canonically “a person and their associates,” as in (\ref{ex:ezzu_basic}). The focal referent\marginnote{Terminology in this section adapted from \href{https://amor.cms.hu-berlin.de/~h2816i3x/Lehre/2007_VL_Typologie/03_Daniel_AssociativePlural.pdf}{Daniel and Moravcsik (2007)}.} (\ie most important) is the marked noun.

\begin{example} \label{ex:ezzu_basic}
    \script Sayanalal oti ezzu Kanyi z-laran.
    \bits sayenal-al ot-i ezzu Kanyi ez=laran
    \gloss {be ignorant}-\tsc{cvb} be-\tsc{prox} \tsc{assoc} \tsc{name} in=this
    \tr For this, Kanyi and his friends won't be much help.
\end{example}

However, the associative can also have a number of idiomatic, context-specific meanings, usually referring to diverse groups.

\begin{example} \label{ex:ezzu_idiom}
    \script ezzu Mázzirran segi s-yiat saska hakra.
    \bits ezzu mázzirran segi su=yiat {saska hakra}
    \gloss \tsc{assoc} ballers want:\tsc{refl} to=steal comeback
    \tr Each and every player wants to pull of the upset.
\end{example}

\section{Gender}
Although nouns traditionally distinguish \emph{common} and \emph{neuter} gender, this has largely become a prescriptive convention. \marginnote{The gender distinction is more common in literature or academia.} Loanwords, especially technical loanwords, are typically assigned neuter gender. Some words only distinguish gender for certain uses or contexts, thus dictionaries typically denote if a given usage is expected to require neuter gender.

\section{Apposition}
Apposition is fairly common due to the small adjective class present in \langname{}. Apposition is used in a number of collocative constructions but is distinct from compounding largely because of stress patterns and morphophonological processes.\marginnote{Compounds can shift stress and also show cross-morpheme sound changes.} Furthermore, appositive nouns shown agreement with the noun they modify, like adjectives.

\subsection{Inalienable possession}
Apposition is a fairly common strategy for possession of all types,\marginnote{Alienable possession via apposition is dispreferred in formal registers.} but inalienably possessed nouns require apposition, as in (\ref{ex:inalien_poss}).

\begin{subexamples}
    \baarucols2
    \ex \label{ex:inalien_poss}
        \script hora kęsa
        \tr soldier's wrist
    \ex
        \script:? hora im-kęsa
        \tr soldier's wrist
\end{subexamples}

The prepositional construction more readily lends itself to a reading indicating inalienable possession, such as “soldier's craftmanship” or “soldier's weaponsmithing.”

% \subsection{Reducing syntactic complexity}
% Appositives are also used as a way of combating heaviness in complex noun phrases. This often occurs in noun phrases with multiple prepositional modifiers, especially when one of the modifiers is \rz{im} “of.” Other prepositions can be reduced this way as well. 

% When multiple prepositional phrases modify a head noun the phrase is syntactically heavy. For example (next example) has three modifiers, an adjective and two prepositional phrases, one with its own adjective.

% % \begin{gloss*}
% %     \begingl
% %         \glpreamble rat Hecik pakné taá im-lalasan ez-helazzaran acoan. \endpreamble
% %         rat[I]
% %         hecik[be.at]
% %         pakné[house\tbs\tsc{dist}]
% %         taá[blue\tbs\tsc{dist}]
% %         im=lalasa-n[of=auntie-\tsc{prox}]
% %         ez=helazzaran[\tsc{prep}=mountains]
% %         acoan[tall]
% %         \glft “I'll be at auntie's blue house near the tall mountains.”
% %     \endgl
% % \end{gloss*}

% To reduce the syntactic load of the sentence, a speaker might instead render it as (next example). The phrase is clearly appositional; \rz{lalasa} switches from proximal marking (indicating the speaker has met her) to distal marking (to agree with its head noun). 

% % \begin{gloss*}
% %     \begingl
% %         \glpreamble rat Hecik pakné taá lalasá ez-helazzaran acoan. \endpreamble
% %         rat[I]
% %         hecik[be.at]
% %         pakné[house\tbs\tsc{dist}]
% %         taá[blue\tbs\tsc{dist}]
% %         lalasá[auntie\tbs\tsc{dist}]
% %         ez=helazzaran[\tsc{prep}=mountains]
% %         acoan[tall]
% %         \glft “I'll be at auntie's blue house near the tall mountains.”
% %     \endgl
% % \end{gloss*}

% The new sentence has fewer function words and less disparate inflections, reducing some of its complexity.

\section{Phrasal syntax}
Noun phrases are predominantly head-initial. Generally speaking, syntactically simpler constituents occur before more syntactically complex ones.

\begin{example}
    head → adjective → number → appositive → prepositional phrase
\end{example}

\setchapterpreamble[u]{\margintoc}
\chapter{Pronouns}
Pronouns are morphologically and syntactically similar to other nouns---they share inflectional patterns and can be modified by adjectives. The notable differences are that pronouns rarely inflect for deixis, and some appositive constructions are ungrammatical.

\begin{kaobox}[frametitle=\sc todo:]
Settle on pronominal forms---right now it's \rz{rat} \tsc{1}, \rz{a(f)} \tsc{2}, \rz{sec} \tsc{3c}, and \rz{moc} \tsc{3n}. There's some isogloss map about whether \tsc{2} is \rz{a} or \rz{af}.
\end{kaobox}

\paragraph{Possession}
Like nominal possession, pronominal possession can be expressed either through apposition or prepositional phrases. \marginnote{Formal registers prefer the prepositional construction.} Thus both \rz{lar im-sec} and \rz{lar sec} can mean “his thing.” However, apposition is more common in isolation for pronouns than it is for nouns.