\setchapterpreamble[u]{\margintoc}
\chapter{Verbs}
The \langname{} verb phrase is more inflectional than the noun phrase, but periphrastic constructions handle most temporal marking, with inflection largely reserved for agreement. Because verbs have fixed valency, prepositional adjuncts, periphrastic constructions, and dummy nouns are common.

\section{Stems}
% Most verbs are \rz{t}-stem verbs, although some verbs are \rz{m}-stem verbs and some are \rz{r}-stem verbs. Verbs conjugation is largely predictable via morphosyntactic processes.

% \begin{kaobox}[frametitle=\sc todo:]
% 	Figure out why the stem differences exist---long-forgotten derivation? maybe some kind of historical aspect stuff that got baked in? Maybe it's just like forever ago?
% \end{kaobox}

\subsection{\rz{t}-stems}
\rz{T}-stems are fairly predictable, although there is some affrication of /t/ in certain forms.

\begin{table}[h] \centering
	\begin{tabular}{l|ccc}
		\toprule
		& \sc\bf gen & \sc\bf prox & \sc\bf dist \\ 
		\midrule
		\sc\bf → 0 & \rzc\it het & \rzc\it hetik & \rzc\it hét \\
		\sc\bf → 1 & \rzc\it hetta & \rzc\it hettik & \rzc\it hattá\\
		\sc\bf → 2 & \rzc\it hetca & \rzc\it hetcik & \rzc\it hatcá \\
		\sc\bf → 3c & \rzc\it hec & \rzc\it hecik & \rzc\it hacá \\
		\sc\bf → 3n & \rzc\it  hetna & \rzc\it hetnik & \rzc\it hatná \\
		\sc\bf 3n → & \rzc\it hecz & \rzc\it heczik & \rzc\it haczá\\
		\sc\bf refl & \rzc\it hetka & \rzc\it hetkik & \rzc\it hatká \\
		\bottomrule
	\end{tabular}
	\caption{Conjugation of \rz{t}-stem \rz{het}}
\end{table}

\subsection{\rz{r}-stems}
\rz{R}-stem verbs are generally predictable in most forms. However, they share morphology for both neuter forms agent and patient.

\begin{table}[h] \centering
	\begin{tabular}{l|ccc}
		\toprule
		& \sc\bf gen & \sc\bf prox & \sc\bf dist \\ 
		\midrule
		\sc\bf → 0 & \rzc\it par & \rzc\it parik & \rzc\it pár \\
		\sc\bf → 1 & \rzc\it parta & \rzc\it partik & \rzc\it partá \\
		\sc\bf → 2 & \rzc\it parca & \rzc\it parcik & \rzc\it parcá \\
		\sc\bf → 3c & \rzc\it pars & \rzc\it parsik & \rzc\it parsá \\
		\sc\bf 3n & \rzc\it paz & \rzc\it pazzik & \rzc\it páz \\
		\sc\bf refl & \rzc\it parka & \rzc\it parkik & \rzc\it parká \\
		\bottomrule
	\end{tabular}
	\caption{Conjugation of \rz{r}-stem \rz{par}}
\end{table}

\subsection{\rz{m}-stems}
\rz{M}-stems are notable for widespread voicing caused by /m/ clusters.

\begin{table}[h] \centering
	\begin{tabular}{l|ccc}
		\toprule
		& \sc\bf gen & \sc\bf prox & \sc\bf dist \\ 
		\midrule
		\sc\bf → 0 & \rzc\it sem & \rzc\it semik & \rzc\it sém \\
		\sc\bf → 1 & \rzc\it sed & \rzc\it sedik & \rzc\it sadá \\
		\sc\bf → 2 & \rzc\it sej & \rzc\it sejik & \rzc\it sajá \\
		\sc\bf → 3c & \rzc\it sems & \rzc\it semsik & \rzc\it samsá \\
		\sc\bf → 3n & \rzc\it semna & \rzc\it semnik & \rzc\it samná \\
		\sc\bf 3n → & \rzc\it semz & \rzc\it semzik  & \rzc\it samzá \\
		\sc\bf refl & \rzc\it seg & \rzc\it segik & \rzc\it sagá \\
		\bottomrule
	\end{tabular}
	\caption{Conjugation of \rz{m}-stem \rz{sem}}
\end{table}

\section{Agreement}
Verbs agree with both the deictic position of the subject noun and the person of the two least oblique arguments of transitive verbs. Only the sole finite verb of a clause bears agreement.

\subsection{Polypersonal}
Transitive verbs exhibit polypersonal agreement via a suffix to the verb root. Intransitive verbs don't require person marking, but it can be used for emphasis or clarification; in such cases, either the reflexive or \tsc{3c} patient morpheme is used.

\begin{table}[h] \centering
\begin{tabular}{cc|cccc}
	& & \multicolumn{4}{c}{\textbf{Patient}} \\
	& & \textbf{1} & \textbf{2} & \textbf{\tsc{3c}} & \textbf{\tsc{3n}} \\ \midrule
	\multirow{5}{*}{\textbf{Agent}} & \textbf{1} & - & c\cellcolor{purple!25} & s\cellcolor{yellow!25} & n\cellcolor{blue!25} \\
	& \textbf{2} & t\cellcolor{green!25} & - & s\cellcolor{yellow!25} & n\cellcolor{blue!25} \\
	& \textbf{\tsc{3c}} & t\cellcolor{green!25} & s\cellcolor{yellow!25} & s\cellcolor{yellow!25} & n\cellcolor{blue!25} \\
	& \textbf{\tsc{3n}} & n\cellcolor{blue!25} & z\cellcolor{red!25} & z\cellcolor{red!25} & z\cellcolor{red!25} \\
	& \textbf{\tsc{refl}} & \multicolumn{4}{c}{k\cellcolor{black!25}} \\
\end{tabular}
\caption{Person agreement}
\end{table}

For the most part verb agreement is actually patient agreement, but some isolated cases are actually polypersonal. Because so many agreement suffixes share the same form, \langname{} is only occasionally pro-drop.

\subsection{Deictic}
Generic noun forms do not have agreement morphemes, but proximal nouns demand the verbal suffix \rz{-ik} and distal nouns shift stress to the final syllable.\marginnote{The stress shifting on both nouns and verbs derives from the same, now elided, morpheme.} These suffixes are attached after polypersonal agreement suffixes.

\begin{kaobox}[frametitle=\sc todo:]
	Rework agreement with distal nouns: it should be more interesting, perhaps shifting stress in some places but not others. For example \tsc{3c.pat} forms might not shift stress because [s] is pretty cluster-happy.
\end{kaobox}

\section{Negation}
Negation is handled with a verbal suffix \rz{-res} \marginnote{The affix's position is from its origin as an auxiliary which bore agreement.} which occurs before agreement affixes. Verbal negation is the primary form of negation in \langname{}, so the forms are rather common. It has a number of allomorphs shown in Table \ref{tab:negative_verb_suffixes}. 

\begin{table}[h] \centering
	\begin{tabular}{l|ccc}
		\toprule
		& \sc\bf gen & \sc\bf prox & \sc\bf dist \\ 
		\midrule
		\sc\bf → 0/2/3c & \rzc\it -ras & \multirow{2}{*}{\rzc\it -réks} & \rzc\it -rés \\
		\sc\bf → 1 & \rzc\it -rac & & \rzc\it -réc \\
		\sc\bf → 3n & \rzc\it -rans & \rzc\it -régs & \rzc\it -réns \\
		\sc\bf 3n → & \rzc\it -raz & \rzc\it -rékz & \rzc\it -réz \\
		\sc\bf refl & \rzc\it -raks & \rzc\it -rék & \rzc\it -réks \\
		\bottomrule
	\end{tabular}
	\caption{Negative suffixes}
	\label{tab:negative_verb_suffixes}
\end{table}

\marginnote[*-3]{The affix is morphophonemically ⫽ɹes⫽.}

Because of the merger of the intransitive, second-person and third-person patient forms, \langname{} is considered to have assymetrical negation. As a consequence, explicit person marking via pronouns is more common with negated verbs, especially in the proximal, where all common persons shared the same phonetic form. 

\section{Converb}
The converb form of a verb is used for simultaneous action. The converb is commonly used to describe the manner of the main clause, and is also commonly used in periphrastic constructions. The converbial suffix is \rz{-nat}, although some roots select \rz{-os}.

\begin{kaobox}[frametitle=\sc todo:]
	Work more on the converb and flesh this out with examples. Might change the morphophonemic form.
\end{kaobox}

\section{Transitivity}
Transitivity is lexically fixed, but transitive verbs can still be functionally intransitive with the use of dummy objects.

Prescriptive convention holds that verbs exhibit polypersonal agreement with their dummy objects, but speakers commonly omit polypersonal marking in these constructions, as in (\nextx b).

\begin{gloss}
	\a \begingl
		\glpreamble sec Sasamsik tasa. \endpreamble
			sec[\tsc{3c}]
			sesam-s-ik[say-\tsc{3c.p-prox}]
			tasa[letter]
		\glft “He's saying something.”
		\trailingcitation (Formal)
	\endgl
	\a \begingl
		\glpreamble sec Semik lar. \endpreamble
			sec[\tsc{3c}]
			sem-ik[say-\tsc{prox}]
			lar[\tsc{exp}]
		\glft “He's talking.”
		\trailingcitation (Informal)
	\endgl \marginnote[*-3]{Informal constructions often use the more grammaticalized \rz{lar} instead of the dummy collocative.}
\end{gloss}

Not every verb has a collocated intransitive form. For these verbs, periphrastic constructions can also serve as valency-changing operations.

\section{Aspect and mood}
Periphrastic constructions handle most temporal marking in \langname{}, covering aspects and moods. The core verb of the periphrastic construction is the only finite verb of a clause, bearing all agreement, and the semantic verb is demoted to an adjunct in converbial form or as a bare infinitive with an adpositional clitic.

\begin{kaobox}[frametitle=\sc todo:]
	Flesh this section out with more examples and more specific uses.
\end{kaobox}

\subsection{\rz{ot}}
The verb \rz{ot} “be” has two periphrastic constructions, a \emph{perfective} and a \emph{support} construction used for focus-fronting verbs.

\paragraph{Perfective}
In the perfective construction, the semantic verb is demoted to adjunct with the preposition \rz{tę}. The perfective can be used in any time frame, although by default it does have a past-time connotation.

The perfective construction generally used for events that occured over a fixed time frame, especially when a duration is given. In contrast with an unmarked verb, the perfective emphasizes sequences of events or actions done a finite number of times.

\paragraph{Support}
In the support construction, the semantic verb is demoted to adjunct as a converb. This construction is used for fronting verbs for emphasis, and does not convey any aspectual information.

\subsection{\rz{sesam}}
The verb \rz{sesam} “say” has one periphrastic construction, an irrealis.\marginnote{\rz{Sesam} is often shorted to \rz{sem} informally.}

\paragraph{Irrealis}
In the irrealis construction, the semantic verb is demoted to adjunct with the preposition \rz{tę}. The irrealis can be used in any time frame, although by default it does have future-time connotation.

The irrealis is used for all events that the speaker supposes might or might've occured. The construction is very general, and has broad semantic meaning---including conditional, jussive, and optative senses. In contrast with the unmarked verb, the irrealis emphasizes that the situation is not factual, but is hoped or posited to occur or have occured.

\subsection{\rz{nenat}}
The verb \rz{nenat} “crouch” has one periphrastic construction, a subjunctive.

\paragraph{Subjunctive}
In the subjunctive construction, the semantic verb is demoted to adjunct with the preposition \rz{ah}. The subjunctive construction has a more limited scope than the \rz{sesam} irrealis construction, typically expressing counterfactuals or doubt.

\subsection{\rz{het}}
The verb \rz{het} “be at” has two periphrastic constructions, a \emph{passive} and an \emph{imperfective}, the latter typically restricted to narrative contexts.

\paragraph{Passive}
In the passive construction, the semantic verb is demoted to adjunct as a converb. The \rz{het} passive is not a true passive because the verb does not change valency (\ie the \tsc{a}-like argument cannot be omitted). Instead of valency operations, the role of this construction is typically to change verbal agreement, as in (\nextx).

\begin{gloss*}
	\a \ljudge{*} \begingl
		\glpreamble Azzár hossusarsik nassoin.\endpreamble 
			as-zár[foreigner-\tsc{pl.dist}]
			hassusar-s-ik[exalt-\tsc{3c»3c-prox}]
			nassoi-n[king-\tsc{prox}]
		\glft \textit{Intended:} “(I see) the foreigners (I've heard about) praising the king.”
	\endgl
	\a \begingl
		\glpreamble Nassoin hecik azzár hossusarnat.\endpreamble
			nassoi-n[king-\tsc{prox}]
			het-s-ik[be.at\tsc{-3c»3c-prox}]
			as-zár[foreigner-\tsc{pl.dist}]
			hassusar-nat[exalt-\tsc{cvb}]
		\glft “(I see) the king being praised by the foreigners (I've heard about).”
	\endgl
\end{gloss*}

In (\lastx a), the hypothetical speaker intends to mark the verb as proximal to convey direct evidence, but the utterance is ungrammatical because the verb doesn't agree with its subject, \rz{azzár}. To correct this, the construction in (\lastx b) is used, which takes advantage of the passive to mark the verb phrase as proximal.

\par In addition to its role shuffling agreement, the \rz{het} passive can also be used to clarify sentences that become ambiguous due to focus fronting, as in (\nextx).

\begin{gloss*}
	\a \begingl
		\glpreamble Kanyin akakvatcik Arpatan. \endpreamble
			Kanyi-n[\tsc{name-prox}]
			akekvat-s-ik[tag-\tsc{3c»3c-prox}]
			Arpat-n[\tsc{name-prox}]
		\glft “Kanyi tagged Arpat.” \\ \textit{or} “Who Arpat tagged was Kanyi.”
	\endgl
	\a \begingl
		\glpreamble Kanyin hecik Arpatan akakvatnat. \endpreamble
			Kanyi-n[\tsc{name-prox}]
			het-s-ik[be.at\tsc{-3c»3c-prox}]
			Arpat-n[\tsc{name-prox}]
			akekvat-nat[tag-\tsc{cvb}]
		\glft “Kanyi was tagged by Arpat.”
	\endgl
\end{gloss*}

In (\lastx a), it's not clear if Kanyi is the semantic agent or a semantic patient that's been fronted for focus. Without context, both interpretations are grammatical. The use of the passive in (\lastx b) is less ambiguously interpreted, almost always meaning that Arpat was the semantic agent.

\subsection{\rz{pit}}
\begin{kaobox}[frametitle=\sc todo:]
    Rework the \rz{pit} passive, which currently doesn't make a lot of sense---how is it demoting stuff, when semantically you'd expect “hold” to be transitive? Maybe \rz{pit} means something else, maybe it will demote in a different way, maybe the valency is weirder...
\end{kaobox}

The verb \rz{pit} “hold” has one periphrastic construction, the mediopassive. \marginnote[*-2]{Unlike most other periphrastic verbs, \rz{pit} is rarely used outside periphrasis, having largely been replaced by \rz{akrar}.}

\paragraph{Mediopassive}
In the mediopassive construction, the semantic verb is demoted to adjunct with the preposition \rz{ez}. Unlike the \rz{het} passive, the \rz{pit} mediopassive is a true passive; the \tsc{a}-like argument does not appear.