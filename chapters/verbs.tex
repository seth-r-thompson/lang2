\setchapterpreamble[u]{\margintoc}
\chapter{Verbs}
Like the noun phrase, the \langname{} verb phrase is mostly analytic, and inflection is largely reserved for agreement. Other parts of the verb complex are handled by periphrastic constructions, notably including valency operations and TAM.

Verbs typically fall into one of three stems: \rz{t}-stems, \rz{m}-stems and \rz{r}-stems. Of the three, \rz{t}-stems are by far the most common and include both transitive and intransitive verbs. The \rz{m}-stem contains just a few members ... The majority of \rz{r}-stem verbs are intransitive, derived from a historical detransitivizer suffix -\pz{?}.

\begin{kaobox}[frametitle=\sc todo:]
	Figure more about the stems--how does that interact with valency, too? Maybe verbs are a lot more lax with endings?
	Japanese has a dichotomy of \emph{tr} to \emph{in} via -\emph{r}, \emph{in} to \emph{tr} via -\emph{s}, and some roots that reflect both endings. Maybe I can do something like that?
\end{kaobox}

\section{Agreement}
Verbs display agreement along two axes: \emph{deixis} and \emph{person}. Deictic agreement is only with the subject of the clause, but person agreement\marginnote[*-1]{Despite person agreement, \langname{} is rarely pro-drop.} is with both subject and the object, if the verb is transitive. Only the sole finite verb of a clause bears agreement; verbs demoted to adjuncts or arguments are always uninflected.

\subsection{Person}
In old \langname{}, verb agreement was transparently derived from cliticized pronouns. However, sound change reduced agreement endings, leading to synchronic forms that are often actually patient agreement.

\begin{kaobox}[frametitle=\sc todo:]
	How to cover morphophonological stuff? The two main ones are affrication of /t/ + /s z/ and cluster-final /s ts/ merging to [s], spelled \rz{s}.
\end{kaobox}

\begin{margintable} \centering
	\begin{tabular}{cc}
		\toprule
		\bf Morpheme & \bf Meaning \\
		\midrule
		\it -\rz{rc}- & \sc 1.ag \\
		\it -\rz{r}- & \sc 1.pt \\
		\it -\rz{a}- & \sc 2 \\
		\it -\rz{s}- & \sc 3c \\
		\it -\rz{z}- & \sc 3n \\
		\it -\rz{ns}- & \tsc{1}↔\tsc{3n} \\
		\it -\rz{k}- & \sc refl \\
		\bottomrule
	\end{tabular}
	\caption{Summary of person agreement}
	\label{tbl:person}
\end{margintable}

Except in a few outlier cases, agreement is fairly predictable from Table \ref{tbl:person}. Neuter agreement takes precedent over other morphemes when applicable. Furthermore, -\rz{a}- only appears on intransitive verbs; for transitive verbs, the appropriate agreement with the other argument is used instead. Finally, -\rz{r}- is used for intransitives,\marginnote{1st-person agreement is ergative as an accident of sound change.} not the agent form.

When attached to a finite verb without deictic agreement, person agreement morphemes are realized as short: unlike many other affixes, they do not shift stress. This realization preserves old stress patterns before the morphemes were reduced. Because of this quirk, some scholars argue that person agreement morphemes are actually clitics to which deictic affixes attach.

\begin{kaobox}[frametitle=\sc todo:]
	Do I want to keep this or say they got analogized? Maybe there's some weird interactions with the stress-fixing rules I've been cooking up?
\end{kaobox}

Because \langname{} allows frequent fronting, a clause's agent and patient can become ambiguous when the agreement morphemes are not sufficient. This especially occurs with common and neuter agreement. A mediopassive\marginnote{The \rz{het} construction is not a true passive: the verb does not change valency (\ie the \tsc{a}-like argument can't be omitted).} construction formed with \rz{het} can be used to clarify such instances. In the mediopassive, the semantic verb is demoted to adjunct as a converb.

\begin{gloss*}
	\a \begingl
		\glpreamble Kanyi akvací Arpat. \endpreamble
			Kanyi[\tsc{name}]
			akvat-s-í[tag-\tsc{3c-prox}]
			Arpat[\tsc{name}]
		\glft “Kanyi tagged Arpat.” \\ \textit{or} “Who Arpat tagged was Kanyi.”
	\endgl
	\a \begingl
		\glpreamble Kanyi hací Arpat akvetnal. \endpreamble
			Kanyi[\tsc{name}]
			het-s-í[be\tsc{-3c-prox}]
			Arpat[\tsc{name}]
			akvatnal[tagging]
		\glft “Kanyi was tagged by Arpat.”
	\endgl
\end{gloss*}

In (\lastx a), it's not clear if Kanyi is the tagger or the focus-fronted taggee; both interpretations are grammatical. The use of the passive in (\lastx b) is less ambiguously interpreted, always meaning that Arpat is the semantic agent.

\subsection{Deixis}
Generic noun forms do not have agreement morphemes, but proximal nouns demand the verbal suffix -\rz{i} and distal nouns shift stress to the final syllable.\marginnote{The stress shift is a remnant of an elided morpheme.} These suffixes are attached after person agreement suffixes. The proximal suffix -\rz{i} is often stressed itself.

\begin{kaobox}[frametitle=\sc todo:]
	Same question as earlier: do I like the stress rules for this affix? Or the way this next one preserves things?
\end{kaobox}

Due to historical stress rules, the distal agreement suffix is more fusional for some select verb forms. Although the rime was lost from the common and neuter agreement suffixes in other forms, it remains in the distal form.

\begin{margintable}[*-5] \centering
	\begin{tabular}{cc}
		\toprule
		\bf Generic & \bf Distal \\
		\midrule
		\it -\rz{s} & \it -\rz{séc} \\
		\it -\rz{z} & \it -\rz{zóc} \\
		\bottomrule
	\end{tabular}
	\caption{Deictic forms of agreement}
\end{margintable}

Deictic agreement requires verbs to encode the same evidentiality as the subject noun phrase, but this may be unfelicitious in some contexts. The \rz{het} construction can also be used to change verbal agreement.

\begin{gloss*}
	\a \ljudge{*} \begingl
		\glpreamble Azzár hassusarsí nassoin.\endpreamble 
			as-zár[foreigner-\tsc{pl.dist}]
			hassusar-s-í[exalt-\tsc{3c-prox}]
			nassoi-n[king-\tsc{prox}]
		\glft \textit{Intended:} “(I see) the foreigners (I've heard about) praising the king.”
	\endgl
	\a \begingl
		\glpreamble Nassoin hací azzár hossusarnal.\endpreamble
			nassoi-n[king-\tsc{prox}]
			het-s-ik[be\tsc{-3c-prox}]
			as-zár[foreigner-\tsc{pl.dist}]
			hossusarnal[exalting]
		\glft “(I see) the king being praised by the foreigners (I've heard about).”
	\endgl
\end{gloss*}

In (\lastx a), the speaker intends to mark the verb as proximal to convey direct evidence, but the utterance is ungrammatical because the verb doesn't agree with its subject, \rz{azzár}. To correct this, the construction in (\lastx b) is used, which takes advantage of the passive to mark the verb phrase as proximal.

\section{Converb}
The converb form of a verb is used for simultaneous action. The converb is commonly used to describe the manner of the main clause, and is also commonly used in periphrastic constructions. The converbial suffix is \rz{-nat}, although some roots select \rz{-os}.

\begin{kaobox}[frametitle=\sc todo:]
	Work more on the converb and flesh this out with examples. Might change the morphophonemic form.
\end{kaobox}

\section{Transitivity}
Transitivity is lexically set. There are three valency classes a verb can fall into: \emph{transitive}, \emph{intransitive}, and \emph{pseudo-transitive}. Transitive verbs always have two arguments and intransitive verbs always have one.\marginnote{All valency classes allow a number of optional but often collocated adjuncts, introduced as prepositional phrases or converbs.} Pseudo-transitive verbs also take more than one argument, but are morphologically intransitive, and as such their additional argument is a prepositional phrase that cannot be ommitted. 

Non-finite verbs of each type do not have fixed valence and can omit all arguments. In some periphrastic constructions, those arguments are still required by the new finite verb. However, other constructions, and general adjectival or adverbial use, often appear without overt arguments.

There are few methods to decrease a verb's valency, which is usually done when an argument is sufficiently clear from context. In contrast, there are no methods to increase a verb's valency.

The most common method of valency reduction is dummy objects. Most transitive verbs have a collocated dummy object.\marginnote{Some verbs have multiple collocations for different senses.} Prescriptive convention holds that verbs exhibit person agreement with these objects, but in speech these verbs are often treated as morphologically intransitive, as in (\nextx b). Although colloquial, this phenomenon points to further grammaticalization of dummy objects.

\begin{gloss}
	\a \begingl
		\glpreamble a Sesamsí tasa. \endpreamble
			a[2]
			sesam-s-í[say-\tsc{3c-prox}]
			tasa[letter]
		\glft “You're saying something.”
		\trailingcitation (Formal)
	\endgl
	\a \begingl
		\glpreamble a Samai lar. \endpreamble
			a[2]
			sem-a-i[say-\tsc{2-prox}]
			lar[\tsc{exp}]
		\glft “You're talking.”
		\trailingcitation (Informal)
	\endgl \marginnote[*-5]{Informally, the more grammaticalized \rz{lar} is more common than a collocated dummy noun.}
\end{gloss}

For verbs that lack a distinct collocated intransitive form, or for certain pragmatic reasons, a periphrastic construction can also serve as a valency-changing operation. The intransitive copula \rz{ot} is the most common, demoting the semantic verb to adjunct as a converb.

\begin{gloss}
	\a \begingl
		\glpreamble *Nassoin kęstací. \endpreamble
			nassoin[the king]
			kęstací[leads]
		\glft (Intended) “The king's in charge.”
	\endgl
	\a \begingl
		\glpreamble Nassoin oc kąstetnal. \endpreamble
			nassoin[the king]
			oc[is]
			kąstetnal[leading]
		\glft “The king's in charge.”
	\endgl \marginnote[*-8]{Ungrammatical; \rz{kęstat} is a transitive verb.}
\end{gloss}

Although rarer than object omission, subject omission is accomplished through the \rz{pit} passive. The passive construction with \rz{pit} promotes the object to subject and demotes the semantic verb to an adjunct with \rz{ez}.

\section{Negation}
Negation can be handled in multiple ways. The typical method is a periphrastic construction with the verb \rz{rek}. Other methods include the use of discourse particles \dots

\begin{kaobox}[frametitle=\sc todo:]
	Write the section about which particles can convey negation, when, and how etc.
\end{kaobox}

\section{Aspect and mood}
\subsection{Perfective}
The perfective construction uses the instransitive copula \rz{ot}, demoting the semantic verb to adjunct with the preposition \rz{tę}. The perfective can be used in any time frame, although by default it does have a past-time connotation.

The perfective construction generally used for events that occured over a fixed time frame, especially when a duration is given. In contrast with an unmarked verb, the perfective emphasizes sequences of events or actions done a finite number of times.

\subsection{Irrealis}
The imperfective construction uses \rz{sem} “say, want,” demoting the semantic verb to adjunct with the preposition \rz{tę}. The irrealis can be used in any time frame, although by default it does have future-time connotation.

The irrealis is used for all events that the speaker supposes might or might've occured. The construction is very general, and has broad semantic meaning---including conditional, jussive, and optative senses. In contrast with the unmarked verb, the irrealis emphasizes that the situation is not factual, but is hoped or posited to occur or have occured.

\subsection{Subjunctive}
The subjunctive construction uses \rz{nenat} “crouch,” demoting the semantic verb to adjunct with the preposition \rz{ah}. The subjunctive construction has a more limited scope than the \rz{sem} irrealis construction, typically expressing counterfactuals or doubt.