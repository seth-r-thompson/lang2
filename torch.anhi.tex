\documentclass[fontsize=12pt,twoside=false,numbers=noenddot]{class/kaobook}

\usepackage[aligned, extraglosscmds, nochapters]{class/baarux}
\usepackage{class/baabbrevs}
\usepackage{fontspec}
\usepackage{booktabs}
\usepackage{caption}
\usepackage{subcaption}
\usepackage{multicol}
\usepackage[utf8]{inputenc}

% font
\setmainfont{LibertinusSerif}[Extension=.otf,Path=./fonts/,UprightFont=*-Regular,BoldFont=*-Bold,ItalicFont=*-Italic,BoldItalicFont=*-BoldItalic]

% links
\hypersetup{colorlinks=true, linkcolor=black, urlcolor=blue}

% romanization
\newcommand{\rzc}{\color[HTML]{5B2D90}}
\newcommand{\rz}[1]{\textit{\rzc #1}}

% foreign words
\newcommand{\fzc}{\color[HTML]{2D8F5B}}
\newcommand{\fz}[1]{\textit{\fzc #1}}

% proto words
\newcommand{\pzc}{\color[HTML]{8F5B2D}}
\newcommand{\pz}[1]{\textit{\pzc #1}}

% shortcuts for common stuff
\newcommand{\tsc}[1]{\textsc{#1}}
\newcommand{\tit}[1]{\textit{#1}}
\newcommand{\tbf}[1]{\textbf{#1}}
\newcommand{\tbs}{\textbackslash}
\newcommand{\ttd}{\char“~}
\newcommand{\tsps}[1]{\textsuperscript{#1}}
\newcommand{\tsbs}[1]{\textsubscript{#1}}
\newcommand{\tin}{\-\hspace{1cm}}
\renewcommand{\bf}{\bfseries}
\renewcommand{\it}{\itshape}
\renewcommand{\sc}{\scshape}
\newcommand{\mr}[1]{\multirow{2}{*}{#1}}

% glossing style
\baaruset{script.markup=\rzc\bf, bits.markup=\it, gloss.markup=\normalfont, source.markup=\footnotesize\sc\baaruparens, cellspacing=.5em}

\newenvironment{examples*}{
	\begin{minipage}{164.6mm}%
	\examples%
}{%
	\endexamples%
	\end{minipage}%
}%

\newenvironment{example*}{
	\begin{minipage}{164.6mm}%
	\examples%
	\ex%
}{%
	\endexamples%
	\end{minipage}%
}%

\newcommand{\smoyd}[2]{(\href{#1}{5moyd \#\ignorespaces#2})}
\newbaarulinetype{citation}{smoyd}{\footnotesize\sc\smoyd}

% glossing abbreviations
\baabbrev{kn}{Direct evidence}
\baabbrev{un}{Indirect evidence}
\baabbrev{cmn}{Common gender}
\baabbrev{ntr}{Neuter gender}
\baabbrev{dmy}{Dummy argument}
\baabbrev{ag}{Agent}
\baabbrev{rfl}{Reflexive}
\baabbrev{assc}{Associative plural}
\baabbrev{name}{Name}
\baabbrev{qt}{Quotative particle}

\begin{document}
\mainmatter
\chapter{Passage}
An excerpt from \rz{Paknan z-Helzzŕ}, a novel published a little over a decade ago following a great war. The author explores themes of feminist reclamation in postwar times.

\begin{quote} \rzc\it%
    Yira ensaci Purrés. kai:\marginnote{Jared approached Clarissa. “Tomorrow you will lose the house, so now you become its family."}\\
    \tin t-Zalmi mosac paknan a.\\
    \tin tęlr: a Ota t-pava.\\
    
    \tin tęlr: a Ota t-pava. kai:\marginnote{“Then you will tell the family tomorrow I will lose that house," Bernard replied. “The house is ours. Today, I will tell the family of the loss myself."}\\
    \tin t-Zalmi masetrs pakná rat.\\
    Gerraós segi. kai:\\
    \tin Pakná ezzu rat pitóz.\\
    \tin t-Secya pava semrs rat m-rat z-mosat.\\
    
    Yira saci makra. kai:\marginnote{“I have and will never stop searching," Jared resolved. “I should help you. We'll search for that long lost house and family."}\\
    \tin t-Kiray sec su semrs otr rat.\\
    \tin a Semrs rat t-ensal.\\
    \tin raczr Kirayrs paknan halman u-pava.\\
    
    Purrés segi. kai.\marginnote{“I'll help you," Clarissa replied. “My family is in the mountainous city. Bernard crossed through there.\\It's how we met, after all."}\\
    \tin a Anselrs rat. Pavan ratn z-tasisyan z-helns heci.\\
    \tin Gerraós retasaci tasisyan.\\
    \tin tęlr: ezzu rat Rąssatki h-akvan.\\
    
    Gerraós saci makra. kai:\marginnote{“Let's search for them," Bernard resolved. “Let's help each other get there and search for them."}\\
    \tin Segi t-kiray sezzr.\\
    \tin Segi t-ensal ezzu rat tevi su m-ratosat m-kiray.\\
    
    Yira vezamsi. kai:\marginnote{Jared accepts. “We will journey to the valley and find it."}\\
    \tin t-Akvan ahkamr raczr; solr: Lar kirayrs raczr.\\
\end{quote}

The book will go on to become a literary classic. Following the pseudorealist tradition which emerged during the war, the plot is deceptively mundane; the story is told entirely through the conversations between the three characters. 

\chapter{Commentary}
% author = Sia Cunnakra
% My commentary will explore both the themes of the passage and the grammar and translation decisions I made while writing it.

When I was presented with Willmagnify's torch, I was struck by the inclusion of this scene-setting blurb: 

\begin{quote} \it
An extract from a Kinvalic \textnormal{mitrâ} (book, novel), describing the adventures of the literary hero Kité. In a mythical past, Kité has wandered far and wide. Now in the presence of Nhata and Kahatiʔ, he makes a plea.
\end{quote}

I'd never seen something like this before in a relay. Without betraying anything in the passage or taking away from the fun of the game, it set my imagination running, and I knew I had to do something similar for my own torch.

In the \tbf{Légatva} lore there's no tradition of literary heros, at least in the epic or biblical sense. I usually explore more contemporary themes like politics, pop culture, and cosmopolitanism.\marginnote[*-1]{The setting is analogous to our 1930s.} After some brainstorming, I eventually settled on the backdrop being a postwar novel % like those written by F. Scott Fitzgerald or Ernest Hemingway (the ``Lost Generation''). 
The two paragraphs bookending the passage above are my attempt to evoke that same excited imagination in you. \marginnote[*-1]{Although only the first was included with the torch.}

The subtle detail I'm most proud of the novel's title, \rz{Paknan z-Helzzŕ}. The goal was something in the vein of \tit{Atlas Shrugged}---evocative and ear-catching while hiding a deeper meaning. The literal translation is just ``The House in the Mountains," but the use of evidence marking implies that the narrator is familiar with the house yet has never seen the mountains that surround it. What that actually means is left open to the reader.

I also changed the character's names. The protagonist became \rz{Yira}, a type of flower. \tbf{Légatva} culture doesn't really have gender in the same sense as us, but it could read as a man's name.\marginnote{More accurately \rz{Yira} is a name associated with type-A personalities.} The other two characters became \rz{Gerraós} and \rz{Purrés}, meaning ``strong'' and ``wise'' respectively. They're also foreign names, associated with a militant western culture whose homeland is a famous mountain range.

The text itself is mostly faithful to content of the torch, but shifted in tone to feel more literary classic than literary epic. \marginnote{I also added a sentence or two to make it more prose.} The biggest change was structural; instead of nine delineated sentences, I opted for a paragraph structure, broken up by speaker, like you'd find in a novel. %\tbf{Légatva} typesetting could be its own essay, but I tried to make it clear what is speech by indenting it

\chapter{Translation}
\begin{example*}
\script Yira ensaci Purrés. kai:
\bits Yira en- sat -s -i Purrés kai
\gloss NAME in go CMN KN NAME QT
\tr Jared approached Clarissa, saying:
\end{example*}

\begin{example*}
\script t-Zalmi mosac paknan a.
\bits tę= zalmi mosat -s pakna -n a
\gloss to sun lose CMN house KN 2
\tr Tomorrow you lose the house...
\end{example*}

\begin{example*}
\script tęlr: a Ota t-pava.
\bits tęlr a ot -a tę= pava
\gloss thus 2 be 2 to family
\tr ...thus you become it's family.
\end{example*}

\begin{example*}
\script tęlr: a Ota t-pava. kai:
\bits tęlr a ot -a tę= pava kai
\gloss thus 2 be 2 to family QT
\tr Then you will tell the family, saying:
\end{example*}

\begin{example*}
\script t-Zalmi masetrs pakná rat.
\bits tę=zalmi mosat-rc   pakná    rat
\gloss to sun lose 1.AG house:UN 1
\tr Tomorrow I will lose the house.
\end{example*}

\begin{example*}
\script Gerraós segi. kai:
\bits Gerraós sem -k -i kai
\gloss NAME say REFL KN QT
\tr Bernard continues, saying:
\end{example*}

\begin{example*}
\script Pakná ezzu rat pitóz.
\bits pakná ezzu rat pit -óz
\gloss house:UN ASSC 1 be NTR
\tr The house is ours...
\end{example*}

\begin{example*}
\script t-Secya pava semrs rat m-rat z-mosat.
\bits tę= secya pava sem -rc rat im= rat ez= mosat
\gloss to star family say 1.AG 1 of 1 of lose
\tr ... today I myself will tell my family of the loss.
\end{example*}

\begin{example*}
\script Yira saci makra. kai:
\bits Yira sat -s -i makra kai
\gloss NAME go CMN KN shoulder QT
\tr Jared resolves, saying:
\end{example*}

\begin{example*}
\script t-Kiray sec su semrs otr rat.
\bits tę= kiray sec su sem -rc ot -r rat
\gloss to search 3.CMN and say 1.AG be 1 1
\tr I have searched and will search.
\end{example*}

\begin{example*}
\script a Semrs rat t-ensal. 
\bits a sem -rs rat tę= ensal 
\gloss 2 say 1.AG 1 to help 
\tr I should help you. 
\end{example*}

\begin{example*}
\script raczr Kirayrs paknan halman u-pava.
\bits rat -zr kiray -rc pakna -n halma -n u= pava
\gloss 1 PL search 1.AG house KN far KN and family
\tr We'll search for the distant house and family.
\end{example*}

\begin{example*}
\script Purrés segi. kai:
\bits Purrés sem -k -i kai
\gloss NAME say RFL KN QT
\tr Clarissa replies, saying:
\end{example*}

\begin{example*}
\script a Anselrs rat. Pavan ratn z-tasisyan z-helns hatés.
\bits a ensal -rc rat pava -n rat -n ez= tasisya -n ez= helans het -és
\gloss 2 help 1.AG 1 family KN 1 KN in city KN in mountain:KN be 3.UN
\tr I'll help you. My family is in the city in the mountains.
\end{example*}

\begin{example*}
\script Gerraós retasaci tasisyan.
\bits Gerraós retosat -s -i tasisya -n
\gloss NAME cut CMN KN city KN
\tr Gerraós crossed that city.
\end{example*}

\begin{example*}
\script tęlr: ezzu rat Rąssatki h-akvan.
\bits tęlr ezzu rat rąssat -k  -i  ah= akva -n
\gloss thus ASSC 1   meet  RFL KN at valley KN
\tr That's how we all met in the valley.
\end{example*}

\begin{example*}
\script Gerraós saci makra. kai:
\bits Gerraós sat -s  -i  makra    kai
\gloss NAME    go CMN KN shoulder QT
\tr Bernard resolves, saying:
\end{example*}

\begin{example*}
\script Segi t-kiray sezzr.
\bits sem -k  -i  tę= kiray  sec -zr
\gloss say  RFL  KN to search 3  PL
\tr Let's search for them.
\end{example*}

\begin{example*}
\script Segi t-ensal ezzu rat tevi su m-ratosat m-kiray.
\bits sem -k  -i  tę= ensal ezzu rat tevi su  im= ratosat im= kiray
\gloss say RFL KN to help  ASSC 1   each and of cut     of search
\tr Let's help each other with the cross and the search.
\end{example*}

\begin{example*}
\script Yira vezamsi. kai:
\bits Yira vezam -s  -i  kai
\gloss NAME accept CMN KN QT
\tr Kira accepts, saying:
\end{example*}

\begin{example*}
\script t-Akvan ahkamr raczr; solr: Lar kirayrs raczr.
\bits tę= akva  -n  ahkam  -r rat -zr solr lar kiray -rc   rat -zr
\gloss to valley KN journey 1 1   PL and  DMY search 1.AG 1  PL
\tr We will journey to the valley and find it.
\end{example*}

\end{document}