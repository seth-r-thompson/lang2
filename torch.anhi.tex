\documentclass[fontsize=12pt,twoside=false,numbers=noenddot]{class/kaobook}

\usepackage[aligned, nojunctures, nochapters]{class/baarux}
\usepackage{class/baabbrevs}
\usepackage{fontspec}
\usepackage{booktabs}
\usepackage{caption}
\usepackage{subcaption}
\usepackage{multicol}
\usepackage[utf8]{inputenc}

% font
\setmainfont{LibertinusSerif}[Extension=.otf,Path=./fonts/,UprightFont=*-Regular,BoldFont=*-Bold,ItalicFont=*-Italic,BoldItalicFont=*-BoldItalic]

% links
\hypersetup{colorlinks=true, linkcolor=black, urlcolor=blue}

% romanization
\newcommand{\rzc}{\color[HTML]{5B2D90}}
\newcommand{\rz}[1]{\textit{\rzc #1}}

% foreign words
\newcommand{\fzc}{\color[HTML]{2D8F5B}}
\newcommand{\fz}[1]{\textit{\fzc #1}}

% proto words
\newcommand{\pzc}{\color[HTML]{8F5B2D}}
\newcommand{\pz}[1]{\textit{\pzc #1}}

% shortcuts for common stuff
\newcommand{\tsc}[1]{\textsc{#1}}
\newcommand{\tit}[1]{\textit{#1}}
\newcommand{\tbf}[1]{\textbf{#1}}
\newcommand{\tbs}{\textbackslash}
\newcommand{\ttd}{\char“~}
\newcommand{\tsps}[1]{\textsuperscript{#1}}
\newcommand{\tsbs}[1]{\textsubscript{#1}}
\newcommand{\tin}{\-\hspace{1cm}}
\renewcommand{\bf}{\bfseries}
\renewcommand{\it}{\itshape}
\renewcommand{\sc}{\scshape}
\newcommand{\mr}[1]{\multirow{2}{*}{#1}}

% glossing style
\baaruset{script.markup=\rzc\bf, bits.markup=\it, gloss.markup=\normalfont, source.markup=\footnotesize\sc\baaruparens, cellspacing=.5em}

\newenvironment{examples*}{
	\begin{minipage}{164.6mm}%
	\examples%
}{%
	\endexamples%
	\end{minipage}%
}%

\newenvironment{example*}{
	\begin{minipage}{164.6mm}%
	\examples%
	\ex%
}{%
	\endexamples%
	\end{minipage}%
}%

\newcommand{\smoyd}[2]{(\href{#1}{5moyd \#\ignorespaces#2})}
\newbaarulinetype{citation}{smoyd}{\footnotesize\sc\smoyd}

% glossing abbreviations
\baabbrev{kn}{Direct evidence}
\baabbrev{un}{Indirect evidence}
\baabbrev{cmn}{Common gender}
\baabbrev{ntr}{Neuter gender}
\baabbrev{dmy}{Dummy argument}
\baabbrev{assc}{Associative plural}

\begin{document}
\mainmatter
\chapter{Passage}
An excerpt from \rz{Paknan z-Helzzŕ}, a novel published a little over a decade ago following a great war. The author explores themes of feminist reclamation in postwar times.

\begin{quote} \rzc\it%
    Yira ensaci Purrés. kai:\marginnote{Jared approached Clarissa. “Tomorrow you will lose the house, so now you become its family."}\\
    \tin t-Zalmi mosac paknan a.\\
    \tin tęlr: a Ota t-pava.\\
    
    \tin tęlr: a Ota t-pava. kai:\marginnote{“Then you will tell the family tomorrow I will lose that house," Bernard replied. “The house is ours. Today, I will tell the family of the loss myself."}\\
    \tin t-Zalmi masetrs pakná rat.\\
    Gerraós segi. kai:\\
    \tin Pakná ezzu rat pitóz.\\
    \tin t-Secya pava semrs rat m-rat z-mosat.\\
    
    Yira saci makra. kai:\marginnote{“I have and will never stop searching," Jared resolved. “I should help you. We'll search for that long lost house and family."}\\
    \tin t-Kiray sec su semrs otr rat.\\
    \tin a Semrs rat t-ensal.\\
    \tin raczr Kirayrs paknan halman u-pava.\\
    
    Purrés segi. kai.\marginnote{“I'll help you," Clarissa replied. “My family is in the mountainous city. Bernard crossed through there.\\It's how we met, after all."}\\
    \tin a Anselrs rat. Pavan ratn z-tasisyan z-helns heci.\\
    \tin Gerraós retasaci tasisyan.\\
    \tin tęlr: ezzu rat Rąssatki h-akvan.\\
    
    Gerraós saci makra. kai:\marginnote{“Let's search for them," Bernard resolved. “Let's help each other get there and search for them."}\\
    \tin Segi t-kiray sezzr.\\
    \tin Segi t-ensal ezzu rat tevi su m-ratosat m-kiray.\\
    
    Yira vezamsi. kai:\marginnote{Jared accepts. “We will journey to the valley and find it."}\\
    \tin t-Akvan ahkamr raczr; solr: Lar kirayrs raczr.\\
\end{quote}

The book will go on to become a literary classic. Following the pseudorealist tradition which emerged during the war, the plot is deceptively mundane; the story is told entirely through the conversations between the three characters. 

\chapter{Commentary}
% author = Sia Cunnakra
% My commentary will explore both the themes of the passage and the grammar and translation decisions I made while writing it.

When I was presented with Willmagnify's torch, I was struck by the inclusion of this scene-setting blurb: 

\begin{quote}
An extract from a Kinvalic \tit{mitrâ} (book, novel), describing the adventures of the literary hero Kité. In a mythical past, Kité has wandered far and wide. Now in the presence of Nhata and Kahatiʔ, he makes a plea.
\end{quote}

I'd never seen something like this before in a relay. It set the tone in such a joyful way. Without betraying anything in the passage or taking away from the fun of the game, it set my imagination running, and I knew I had to do something similar for my own torch.

In the \tbf{Légatva} lore there's no tradition of literary heros, at least in the epic or biblical sense. I usually explore more contemporary themes like politics, pop culture, and cosmopolitanism.\marginnote[*-1]{The setting is analogous to our 1930s.} After some brainstorming, I eventually settled on the backdrop being a postwar novel. % like those written by F. Scott Fitzgerald or Ernest Hemingway (the ``Lost Generation''). 
The two paragraphs bookending the passage above are my attempt to evoke that same excited imagination in you. \marginnote[*-1]{Although only the first was included with the torch.}

% I had decided to move away from the literary epic motif, but I still wanted to to stay faithful to the spirit of the passage, so I targeted literary classic instead. 

% I'm most proud of the novel's title, \rz{Paknan z-Helzzŕ}. The goal was something in the vein of \tit{Atlas Shrugged}---evocative, ear-catching yet hiding a deeper meaning. The title could idiomatically be translated into English as ...

\end{document}