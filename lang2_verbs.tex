\setchapterpreamble[u]{\margintoc}
\chapter{Verbs}

\begin{kaobox}[frametitle=\sc todo:]
    Like in \emph{§ Nouns}, it would be good to have an overview of stems and inflection patterns before diving into the meaning of the morphemes.
\end{kaobox}

\section{Agreement}
Verbs agree with both the deictic position of the subject noun and the person of the two least oblique arguments of transitive verbs.

\subsection{Polypersonal}
Transitive verbs exhibit polypersonal agreement via a suffix to the verb root. Intransitive verbs don't require person marking, but it can be used for emphasis or clarification; in such cases, either the reflexive or \tsc{3c} patient morpheme is used.

%\begin{table}[h] \centering
%\begin{tabular}{cc|cccc}
%	& & \multicolumn{4}{c}{\textbf{Patient}} \\
%	& & \textbf{1} & \textbf{2} & \textbf{\tsc{3c}} & \textbf{\tsc{3n}} \\ \hline
%	\multirow{5}{*}{\textbf{Agent}} & \textbf{1} & - & c & s & n \\
%	& \textbf{2} & t & - & s & n \\
%	& \textbf{\tsc{3c}} & t & s & s & n \\
%	& \textbf{\tsc{3n}} & n & z & z & z \\
%	& \textbf{\tsc{refl}} & \multicolumn{4}{c}{k} \\
%\end{tabular}
%\caption{Person agreement}
%\end{table}

\begin{table}[h] \centering
\begin{tabular}{cc|cccc}
	& & \multicolumn{4}{c}{\textbf{Patient}} \\
	& & \textbf{1} & \textbf{2} & \textbf{\tsc{3c}} & \textbf{\tsc{3n}} \\ \midrule
	\multirow{5}{*}{\textbf{Agent}} & \textbf{1} & - & c\cellcolor{purple!25} & s\cellcolor{yellow!25} & n\cellcolor{blue!25} \\
	& \textbf{2} & t\cellcolor{green!25} & - & s\cellcolor{yellow!25} & n\cellcolor{blue!25} \\
	& \textbf{\tsc{3c}} & t\cellcolor{green!25} & s\cellcolor{yellow!25} & s\cellcolor{yellow!25} & n\cellcolor{blue!25} \\
	& \textbf{\tsc{3n}} & n\cellcolor{blue!25} & z\cellcolor{red!25} & z\cellcolor{red!25} & z\cellcolor{red!25} \\
	& \textbf{\tsc{refl}} & \multicolumn{4}{c}{k\cellcolor{black!25}} \\
\end{tabular}
\caption{Person agreement}
\end{table}

Because many agreement suffixes share the same form, \langname{} is only occasionally pro-drop.

\subsection{Deictic}
Generic noun forms do not have agreement morphemes, but proximal nouns demand the verbal suffix \rz{-ik} and distal nouns demand the verbal suffix \rz{-ǫ}. These suffixes are attached after polypersonal agreement suffixes.

\section{Negation}
Negation ...

The negative verb suffix is \rz{-res}. \marginnote{The affix's position is from its origin as an auxiliary which bore agreement.} It occurs before agreement affixes.

\section{Converb}
The converb form of a verb is used for simultaneous action. The converb is commonly used to describe the manner of the main clause, and is also commonly used in periphrastic constructions.

The converbial suffix is \rz{-nat}.

\section{Transitivity}
Transitivity is lexically fixed, but transitive verbs can still be functionally intransitive with the use of dummy objects.

Prescriptive convention holds that verbs exhibit polypersonal agreement with their dummy objects, but speakers commonly omit polypersonal marking in these constructions, as in (\nextx b).

\begin{gloss}
	\a \begingl
		\glpreamble sec Sasamsik tasa. \endpreamble
			sec[\tsc{3c}]
			sesam-s-ik[say-\tsc{3c.p-prox}]
			tasa[letter]
		\glft “He's saying something.”
		\trailingcitation (Formal)
	\endgl
	\a \begingl
		\glpreamble sec Semik lar. \endpreamble
			sec[\tsc{3c}]
			sem-ik[say-\tsc{prox}]
			lar[\tsc{exp}]
		\glft “He's talking.”
		\trailingcitation (Informal)
	\endgl \marginnote[*-3]{Informal constructions often use the more grammaticalized \rz{lar} instead of the dummy collocative.}
\end{gloss}

Not every verb has a collocated intransitive form. For these verbs, periphrastic constructions can also serve as valency-changing operations.

\section{Periphraxis}
Periphrastic constructions handle most temporal marking in \langname{}, covering aspects and moods. The core verb of the periphrastic construction is the only finite verb of a clause, bearing all agreement, and the semantic verb is demoted to an adjunct in converbial form or as a bare infinitive with an adpositional clitic.

\subsection{\rz{ot}}
The verb \rz{ot} “be” has two periphrastic constructions, a \emph{perfective} and a \emph{support} construction used for focus-fronting verbs.

\paragraph{Perfective}
In the perfective construction, the semantic verb is demoted to adjunct with the preposition \rz{tę}.

\paragraph{Support}
In the support construction, the semantic verb is demoted to adjunct as a converb.

\subsection{\rz{sesam}}
The verb \rz{sesam} “say” has one periphrastic construction, an irrealis.\marginnote{\rz{Sesam} is often shorted to \rz{sem} informally.}

\paragraph{Irrealis}
In the irrealis construction, the semantic verb is demoted to adjunct with the preposition \rz{tę}. 

\subsection{\rz{nenat}}
The verb \rz{nenat} “” has one periphrastic construction, a subjunctive.

\paragraph{Subjunctive}
In the subjunctive construction, the semantic verb is demoted to adjunct with the preposition \rz{ah}. The subjunctive construction has a more limited scope than the \rz{sesam} irrealis construction, typically expressing counterfactuals or doubt.

\subsection{\rz{het}}
The verb \rz{het} “be at” has two periphrastic constructions, a \emph{passive} and an \emph{imperfective}, the latter typically restricted to narrative contexts.

\paragraph{Passive}
In the passive construction, the semantic verb is demoted to adjunct as a converb. The \rz{het} passive is not a true passive because the verb does not change valency (\ie the \tsc{a}-like argument cannot be omitted). Instead of valency operations, the role of this construction is typically to change verbal agreement, as in (\nextx).

\begin{gloss*}
	\a \ljudge{*} \begingl
		\glpreamble Azzár hossusarsik nassoin.\endpreamble 
			as-zár[foreigner-\tsc{pl.dist}]
			hassusar-s-ik[exalt-\tsc{3c»3c-prox}]
			nassoi-n[king-\tsc{prox}]
		\glft \textit{Intended:} “(I see) the foreigners (I've heard about) praising the king.”
	\endgl
	\a \begingl
		\glpreamble Nassoin hecik azzár hossusarnat.\endpreamble
			nassoi-n[king-\tsc{prox}]
			het-s-ik[be.at\tsc{-3c»3c-prox}]
			as-zár[foreigner-\tsc{pl.dist}]
			hassusar-nat[exalt-\tsc{cvb}]
		\glft “(I see) the king being praised by the foreigners (I've heard about).”
	\endgl
\end{gloss*}

In (\lastx a), the hypothetical speaker intends to mark the verb as proximal to convey direct evidence, but the utterance is ungrammatical because the verb doesn't agree with its subject, \rz{azzár}. To correct this, the construction in (\lastx b) is used, which takes advantage of the passive to mark the verb phrase as proximal.

\par In addition to its role shuffling agreement, the \rz{het} passive can also be used to clarify sentences that become ambiguous due to focus fronting, as in (\nextx).

\begin{gloss*}
	\a \begingl
		\glpreamble Kanyin akakvatcik Arpatan. \endpreamble
			Kanyi-n[\tsc{name-prox}]
			akekvat-s-ik[tag-\tsc{3c»3c-prox}]
			Arpat-n[\tsc{name-prox}]
		\glft “Kanyi tagged Arpat.” \\ \textit{or} “Who Arpat tagged was Kanyi.”
	\endgl
	\a \begingl
		\glpreamble Kanyin hecik Arpatan akakvatnat. \endpreamble
			Kanyi-n[\tsc{name-prox}]
			het-s-ik[be.at\tsc{-3c»3c-prox}]
			Arpat-n[\tsc{name-prox}]
			akekvat-nat[tag-\tsc{cvb}]
		\glft “Kanyi was tagged by Arpat.”
	\endgl
\end{gloss*}

In (\lastx a), it's not clear if Kanyi is the semantic agent or a semantic patient that's been fronted for focus. Without context, both interpretations are grammatical. The use of the passive in (\lastx b) is less ambiguously interpreted, almost always meaning that Arpat was the semantic agent.

\subsection{\rz{pit}}
\begin{kaobox}[frametitle=\sc todo:]
    Rework the \rz{pit} passive, which currently doesn't make a lot of sense--how is it demoting stuff, when semantically you'd expect “hold” to be transitive? Maybe \rz{pit} means something else, maybe it will demote in a different way, maybe the valency is weirder...
\end{kaobox}
The verb \rz{pit} “hold” has one periphrastic construction, the mediopassive. \marginnote[*-2]{Unlike most other periphrastic verbs, \rz{pit} is rarely used outside periphrasis, having largely been replaced by \rz{akrar}.}

\paragraph{Mediopassive}
In the mediopassive construction, the semantic verb is demoted to adjunct with the preposition \rz{ez}. Unlike the \rz{het} passive, the \rz{pit} mediopassive is a true passive; the \tsc{a}-like argument does not appear.
