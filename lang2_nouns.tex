\setchapterpreamble[u]{\margintoc}
\chapter{Nouns}
The \langname{} noun phrase is largely analytic, but nouns do inflect for deixis and number. Nouns have three broad inflection patterns, largely related to the way they inflect for plurality.

\section{Stems}
Nouns are broadly divided into four stems based on their inflectional patterns. Most noun stems are vocalic stems, ending in a vowel. Vocalic stems are usually common gender except for loanwords, which are prescriptively assigned neuter gender. Most native neuter nouns are either \rz{r}-stems or \rz{s}-stems, the latter being more common. \marginnote{\rz{S}-stems can end in any sibilant, typically \rz{-s} but also \rz{-c} or \rz{-z}.} Rarely noun stems will end in other consonants, but these have no shared patterns.

\paragraph{Vocalic stems}
Vocalic stems have fairly regular, agglunative inflection patterns. However, the proximal plural is shortened to \rz{-rran} for most speakers.
\begin{table}[h] \centering
    \begin{tabular}{c|ccc}
        \toprule
        & \bf Generic & \bf Proximal & \bf Distal \\
        \midrule
        \bf \sc sg & \it\rzc metka & \it\rzc metkan & \it\rzc matkó \\
        \bf \sc pl & \it\rzc matkozar & \it\rzc matkorran & \it\rzc matkazár \\
        \bottomrule
    \end{tabular}
    \caption{Inflection of vowel-stem \rz{metka} “bowl”}
    \label{tab:metka_inflection}
\end{table}

\paragraph{Neuter stems}
Neuter stems share a common inflection pattern. For both neuter stems, the proximal surfaces as /on/ instead of /n/. The proximal plural also shortens for neuter stems, but takes the form \rz{-zza-}, influenced by the assibilation morphological process.

\begin{table}[h] \centering
    \begin{tabular}{c|ccc}
        \toprule
        & \bf Generic & \bf Proximal & \bf Distal \\
        \midrule
        \bf \sc sg & \it\rzc retus & \it\rzc ratusan & \it\rzc ratús \\
        \bf \sc pl & \it\rzc ratuzzar & \it\rzc ratuzzan & \it\rzc ratuzzár \\
        \bottomrule
    \end{tabular}
    \caption{Inflection of \rz{s}-stem \rz{retus} “blade”}
\end{table}

\begin{table}[h] \centering
    \begin{tabular}{c|ccc}
        \toprule
        & \bf Generic & \bf Proximal & \bf Distal \\
        \midrule
        \bf \sc sg & \it\rzc pebar & \it\rzc pabaran & \it\rzc pabár \\
        \bf \sc pl & \it\rzc pabazzar & \it\rzc pabazzan & \it\rzc pabazzár \\
        \bottomrule
    \end{tabular}
    \caption{Inflection of \rz{r}-stem \rz{pebar} “garden”}
\end{table}

Although no longer morphologically productive, the endings on \rz{s}-stems and \rz{r}-stems derive from historical derivation processes. Many roots are reflected in both endings, but the shared meaning between them is not always transparent.

\section{Deixis}
Nominal deixis has a variety of uses, including evidentiality, distance, familiarity, and topicality.\marginnote{I had this idea, then found out that, as usual, a natural language had it first. Read \href{http://lingpapers.sites.olt.ubc.ca/files/2020/07/11_ICSNL55_Huijsmans_Reisinger_Matthewson_final.pdf}{Huijsmans, Reisinger, and Matthewson (2020)} for more about the Salishan languages.} Verbs and adjectives exhibit agreement for deictic reference. There are three deictic categories, \emph{generic}, \emph{proximal}, and \emph{distal}.

\subsection{Generic}
The unmarked or dictionary form of a noun is used when the noun is widely understood or well-known, for immaterial referents that cannot be deictically located, or if evidence of the referent is not known. If direct or reported evidence exists, it's felicitous or questionably grammatical to use unmarked form.

\subsection{Proximal}
The proximal form of a noun is used when the speaker is certain, nearby, or familiar with the noun. It can also be used for the conversational topic. This form most commonly denotes direct evidence, meaning the speaker has personal experience with the marked noun. It is marked with the suffix \rz{-n}. \marginnote[*-2]{\rz{-n} is morphophonemically ⫽on⫽, where ⫽o⫽ doesn't surface for vocalic stems.}

% \begin{margintable}
%     \begin{tabular}{cc}
%         \toprule
%         \bf\sc gen & \bf\sc prox \\
%         \midrule
%         \rz{metka} & \rz{metkan} \\
%         \rz{retus} & \rz{ratusan} \\
%         \rz{pebar} & \rz{pabaran} \\
%         \bottomrule
%     \end{tabular}
%     \caption{Proximal inflection for different stems}
% \end{margintable}

\paragraph{Direct evidence}
The canonical meaning of the proximal form is direct evidence, often translated as “I saw.” 

\begin{kaobox}[frametitle=\sc todo:]
    The definiteness constructions probably need to be reworked to square better with (a) the stuff I've learned about definiteness and (b) the use of the deictic forms for topic/focus.
\end{kaobox}

\paragraph{Definiteness}
Proximal forms can be used to describe the definiteness of a referent. This construction is only used for weak, uniqueness-based definiteness (eg. “the Moon”), never for strong, anaphoric definiteness (eg. “the book”). For strong definitess, the noun \rz{sin} “???” is used alongside distal form, as in (\nextx b).

\begin{gloss*}
    \a \begingl
        \glpreamble Nassoin kąstecik su kagęsa su kagę́stapa. \endpreamble
            nassoi-n[king-\tsc{prox}]
            kąstecik[command]
            su[and]
            kagęsa[army]
            su[and]
            kagę́stapa[navy]
        \glft “The king (that we know) commands both army and navy.”
    \endgl
    \a \begingl
        \glpreamble sah ez-Rosąm pít ató sín. \endpreamble
            sah[soon]
            tę=rosąm[\tsc{prep}=cook]
            pít[hold\tbs\tsc{dist}]
            ató[grain\tbs\tsc{dist}]
            sín[???\tbs\tsc{dist}]
        \glft “The rice (that you mentioned) is about to be cooked.”
        \smoyd{https://www.reddit.com/r/conlangs/comments/kck1hi/1381st_just_used_5_minutes_of_your_day/}{1381}
    \endgl
\end{gloss*}

\subsection{Distal}
The distal form of a noun is used when the speaker is uncertain, far, or unfamiliar with the noun. It can also be used for the conversational focus. This form typically denotes indirect evidence, including inference, meaning the speaker has heard of or can make an educated guess about the existence of the marked noun. Reported deixis is marked by shifting stress to the ultimate syllable of the word.

\paragraph{Indirect Evidence}
The prototypical meaning of the distal form is indirect evidence, often translated as “heard about” or “they said.” As in (\nextx), this evidence is encoded into the clause via the subject and the predicate that agrees with it.

\begin{gloss*}
    \begingl
        \glpreamble egi Matkó aczé sém tę-het.\endpreamble
            egi[just]
            matkó[basket\tbs\tsc{dist}]
            aczé[two\tbs\tsc{dist}]
            sém[will:be\tbs\tsc{dist}]
            tę=het[\tsc{prep}=be:at]
        \glft “(She said) there will be just two baskets.”
        \smoyd{https://www.reddit.com/r/conlangs/comments/ic7on8/1314th_just_used_5_minutes_of_your_day/}{1314}
    \endgl
\end{gloss*}

\section{Number}
Nouns inflect morphologically for an additive plural, but there is also a syntactic construction used to form associative plurals. The unmarked form of a noun encodes expected number, \eg \rz{parsa} “eyes” which defaults to a pair and must take a numeral to specify a singulative. 

\begin{figure}[h]
    \centering
    \begin{subfigure}{0.4\textwidth}
        \centering
        \includegraphics[width=\textwidth]{same_vases.jpg}
        \caption{\rz{matkozar}}
    \end{subfigure}
    \begin{subfigure}{0.4\textwidth}
        \centering
        \includegraphics[width=\textwidth]{different_vases.jpg}
        \caption{\rz{ezzar metka}}
    \end{subfigure}
    \caption{Additive vs. associative plurals}
\end{figure}

The primary difference in the two plurals is the composition of the set: additive plurals refer to largely homogenous referents, whereas associative plurals refer to largely heterogenous referents.

\subsection{Additive plurals}
Additive plurals are used for a set of homogenous referents and never heterogenous referents; \eg \rz{matkozar} is “a set of the same (or similar) bowl” and never “a set of diverse bowls.”\marginnote{The second meaning would use the associative plural.}

Additive plurality is indicated through the suffix \rz{-zar}. Morphological marking is optional and a noun can be inferred additively plural from context. \marginnote{Mandatory plural marking is stylistically preferred in formal contexts.} As such marking is less common for small, discrete or easily countable sets or when a referent has been established plural in prior conversation. However, speakers are not always consistent with marking.

\begin{kaobox}[frametitle=\sc todo:]
    These tables probably belong in a different section (perhaps the overview in \emph{§ stems}), not here. Here can just re-hash the relevant bits of the inflection patterns.
\end{kaobox}

Morphonologically, plural marking precedes deictic marking; plural distal nouns have accent placed on the plural suffix, as seen in Table \ref{tab:metka_inflection}. The morphemes ⫽+n͡zaɹ+on⫽ are reduced to /ɹɹon/.

\rz{S}-stem and \rz{r}-stem nouns inflect similarly, except the proximal plural suffix is reduced to /n͡zzon/ instead.

Because the suffixes of \rz{s}-stems and \rz{r}-stems merge in the plural, some minimal pairs are rendered homphones when inflected. To combat this, speakers sometimes employ the use of the word \rz{tevi} “many” as a modifier for the singular form.

When number is specified with a numeral, the noun is not marked for plurality, as in (\nextx). This is another strategy to combat homophony.

\begin{gloss}
    \begingl
        \glpreamble vęci ocza \endpreamble
            vęci[mercenary]
            ocza[two]
        \glft “two mercenaries”
        \trailingcitation (cf. \rz{vącizar} “mercenaries”)
    \endgl
\end{gloss}

\subsection{Associative plurals}
Unlike the additive plural, the associative plural is not marked morphologically. The periphrastic construction \rz{...} conveys the associative meaning. 

\begin{kaobox}
Need to figure out the specific morpheme and/or construction. Could be \rz{ezzar}, \rz{ezzu}, a longer construction \dots
\end{kaobox}

The associative is used for a set of heterogenous referents. For animate (especially human) referents, the meaning is typically “a person and their associates,” as in (\nextx). The focal referent\marginnote{Terminology in this section adapted from \href{https://amor.cms.hu-berlin.de/~h2816i3x/Lehre/2007_VL_Typologie/03_Daniel_AssociativePlural.pdf}{Daniel and Moravcsik (2007)}.} (\ie most important) is the marked noun.

\begin{gloss}
    \begingl
        \glpreamble Sayanarnat otik ezzu Kanyi ez-laran. \endpreamble
            sayenar-nat[be.ignorant-\tsc{cvb}]
            ot-ik[be-\tsc{prox}]
            ezzu[\tsc{assoc}]
            Kanyi[\tsc{name}]
            ez=lar-n[\tsc{prep=expl-prox}]
        \glft “For this, Kanyi and his friends won't be much \emph{help}.”
    \endgl
\end{gloss}

The associative can also have a number of context-specific meanings, usually referring to diverse groups.

% \pex 
% \a \begingl
% \glpreamble ezzu Akakvazi rematt \endpreamble
% \glft “The diverse group of tag players entertained us ...”
% \endgl 
% % \a \begingl
% % \glft “My grandfather painted a wide variety of paintings.”
% % \endgl
% \xe

\section{Gender}
Some archaic nouns distinguish \emph{common} and \emph{neuter} gender, although this is largely a prescriptive convention. \marginnote{The gender distinction is more common in literary, academic, or scientific writings.} Loanwords, especially technical loanwords, are typically assigned neuter gender. Some words only distinguish gender for certain uses or contexts, thus dictionaries typically denote if a given usage is expected to require neuter gender.

\section{Pronouns}
Prounouns are morphologically and syntactically similar to nouns \dots

\begin{kaobox}[frametitle=\sc todo:]
Settle on pronominal forms---right now it's \rz{rat} \tsc{1}, \rz{a(f)} \tsc{2}, \rz{sec} \tsc{3c}, and \rz{moc} \tsc{3n}. There's some isogloss map about whether \tsc{2} is \rz{a} or \rz{af}.
\end{kaobox}