\documentclass[fontsize=12pt,twoside=false,numbers=noenddot]{kaobook}

\usepackage{styles/mdftheorems}
\usepackage{styles/environments}
\usepackage{styles/kaobiblio}
\usepackage{fontspec}
\usepackage{expex}
\usepackage{booktabs}
\usepackage{caption}
\usepackage{subcaption}
\usepackage[utf8]{inputenc}

%--- METADATA ---%
\newcommand{\langname}{\textbf{lang2}}
\title[A Grammar of \langname{}]{\huge \langname{}}
\subtitle{grammar of a constructed language}
\author[kilenc]{\large Seth Thompson \\ \small \textit{aka} kilenc}
\date{\small \today}

%--- FORMATTING ---%
% font
\setmainfont{LibertinusSans}[Extension = .otf , Path = ./ , UprightFont = *-Regular-Custom , BoldFont = *-Bold-Custom , ItalicFont = *-Italic-Custom]
% \setmainfont{LibertinusSans}[Extension = .otf , Path = ./ , UprightFont = *-Regular-Custom , BoldFont = *-Bold-Custom , ItalicFont = *-Italic-Custom , BoldItalicFont = *-Regular-Custom ,BoldItalicFeatures={FakeBold=3,FakeSlant=0.3} ]
% \setmainfont{Brill Roman} % this might be better than Libertinus but the numbers aren't great
% \setmainfont{Libertinus Serif}
\setmonofont{Iosevka}[SizeFeatures = {Size = 10}]
\renewcommand{\bf}{\bfseries}

% graphics
\graphicspath{{/}{images/}} % image paths

% links
\hypersetup{colorlinks=true, linkcolor=black, urlcolor=blue}

% gloss
\definelingstyle{gloss}{glstyle=nlevel, everyglpreamble=\bfseries\color{BrickRed}, aboveglftskip=-8pt, belowglpreambleskip=-8pt, aboveexskip=0pt, belowexskip=-4pt}
\definelingstyle{widegloss}{glstyle=nlevel, everyglpreamble=\bfseries\color{BrickRed}, aboveglftskip=0pt,belowglpreambleskip=0pt, aboveexskip=8pt, belowexskip=4pt}
\lingset{lingstyle=gloss}

\newenvironment{gloss*}{
	\begin{minipage}{164.6mm}%
	\pex[lingstyle=widegloss]%
}{%
	\xe%
	\end{minipage}%
}%

%--- COMMANDS ---%
% romanization
% maybe I should go for MidnightBlue?
% and \fz was \textcolor[HTML]{008086}
\newcommand{\rz}[1]{\textcolor{BrickRed}{\textit{#1}}}
\newcommand{\rzbf}[1]{\textcolor{BrickRed}{\textbf{#1}}}
\newcommand{\rzit}[1]{\textcolor{BrickRed}{\textit{#1}}}
\newcommand{\fz}[1]{\textcolor[HTML]{008086}{\textbf{#1}}}
\newcommand{\rzbrkt}[1]{\textcolor{BrickRed}{⟨\textbf{#1}⟩}}

% glossing shortcuts
\newcommand{\gsc}[1]{\textsc{#1}}
\newcommand{\gbs}{\textbackslash}
\newcommand{\gtd}{\char“~}

% ipa shortcut
\newcommand{\textss}[1]{\textsuperscript{#1}}

% ref shortcuts
\newcommand{\smoyd}[2]{\trailingcitation{\footnotesize \href{#1}{(\gsc{5moyd \#{}#2})}}}

% chapter shortcut
\newcommand{\chaptertoc}[1]{\setchapterpreamble[u]{\margintoc}\chapter{#1}}

\begin{document}

\frontmatter % Denotes the start of the pre-document content, uses roman numerals

\KOMAoptions{twoside=semi}
\maketitle
\KOMAoptions{twoside=false}

% toc formatting
\setlength{\textheight}{23cm} % Manually adjust the height of the ToC pages
\etocstandarddisplaystyle % "toc display" as if etoc was not loaded
\etocstandardlines % toc lines as if etoc was not loaded

\tableofcontents % Output the table of contents

\pagelayout{margin}
\setchapterstyle{kao} % Choose the default chapter heading style

\setchapterpreamble[u]{\margintoc}
\chapter{Introduction}
\section{Origins}
\langname{} is an \emph{a priori} artlang originally conceived to fulfill a speedlang challenge, then modified to handle a relay, then finally molded into its own project. It aims to be smarter about phonology, morphosyntax, pragmatics and semantics than the average conlang through careful and extensive  study of literature. When naturalism conflicts with aesthetic, however, aesthetic will be prioritized.

\section{Goals}
A primary way to grow this language and develop new ideas will be through translation of poetry and scientific journals. I hope that these will push the limits of my syntactical rules while also developing an interesting corpus. I'll also try to use \gsc{5moyd}s and hopefully at some point a journal to expand my ability to `speak' the language and get an intuitive sense for what constructions it prefers.

As I develop this conlang, my goals will be to prioritize analytic constructions (like periphraxis) over morphological ones, write in-depth documentation of how information structure manifests in the language (especially with regards to syntax), and ...

\section{Lore}
\langname{} is set in a worldbuilding project I create as a hobby. The language is an isolate spoken on the eastern coast of a penininsula, once the language of citystates and now a common language throughout a number of coastal countries. It is heavily influenced by East Cape, the \textit{de jure} language of the peninsula. The world the speakers know is analogous to our 1930s.

\mainmatter % Denotes the start of the main document content, resets page numbering and uses arabic numbers

\pagelayout{wide}
\part{Phonology}
\pagelayout{margin}

\chaptertoc{Overview}
... some stuff about typology here ...

\input{lang2_phono.tex}

\pagelayout{wide}
\part{Morphosyntax}
\pagelayout{margin}

\chaptertoc{Overview}
... some stuff about typology here ... there should also be a section on fronting etc ...

\section{Word classes}
There are 3 primary word classes: nouns, verbs, and adpositions. There are also a limited number of adjectives and discourse particles.

\setchapterpreamble[u]{\margintoc}
\chapter{Nouns}
The \langname{} noun phrase is largely analytic, but nouns do inflect for deixis and number. Nouns have three broad inflection patterns, largely related to the way they inflect for plurality.

\section{Stems}
Nouns are broadly divided into four stems based on their inflectional patterns. Most noun stems are vocalic stems, ending in a vowel. Vocalic stems are usually common gender except for loanwords, which are prescriptively assigned neuter gender. Most native neuter nouns are either \rz{r}-stems or \rz{s}-stems, the latter being more common. \marginnote{\rz{S}-stems can end in any sibilant, typically \rz{-s} but also \rz{-c} or \rz{-z}.} Rarely noun stems will end in other consonants, but these have no shared patterns.

\paragraph{Vocalic stems}
Vocalic stems have fairly regular, agglunative inflection patterns. However, the proximal plural is shortened to \rz{-rran} for most speakers.
\begin{table}[h] \centering
    \begin{tabular}{c|ccc}
        \toprule
        & \bf Generic & \bf Proximal & \bf Distal \\
        \midrule
        \bf \sc sg & \it\rzc metka & \it\rzc metkan & \it\rzc matkó \\
        \bf \sc pl & \it\rzc matkozar & \it\rzc matkorran & \it\rzc matkazár \\
        \bottomrule
    \end{tabular}
    \caption{Inflection of vowel-stem \rz{metka} “bowl”}
    \label{tab:metka_inflection}
\end{table}

\paragraph{Neuter stems}
Neuter stems share a common inflection pattern. For both neuter stems, the proximal surfaces as /on/ instead of /n/. The proximal plural also shortens for neuter stems, but takes the form \rz{-zza-}, influenced by the assibilation morphological process.

\begin{table}[h] \centering
    \begin{tabular}{c|ccc}
        \toprule
        & \bf Generic & \bf Proximal & \bf Distal \\
        \midrule
        \bf \sc sg & \it\rzc retus & \it\rzc ratusan & \it\rzc ratús \\
        \bf \sc pl & \it\rzc ratuzzar & \it\rzc ratuzzan & \it\rzc ratuzzár \\
        \bottomrule
    \end{tabular}
    \caption{Inflection of \rz{s}-stem \rz{retus} “blade”}
\end{table}

\begin{table}[h] \centering
    \begin{tabular}{c|ccc}
        \toprule
        & \bf Generic & \bf Proximal & \bf Distal \\
        \midrule
        \bf \sc sg & \it\rzc pebar & \it\rzc pabaran & \it\rzc pabár \\
        \bf \sc pl & \it\rzc pabazzar & \it\rzc pabazzan & \it\rzc pabazzár \\
        \bottomrule
    \end{tabular}
    \caption{Inflection of \rz{r}-stem \rz{pebar} “garden”}
\end{table}

Although no longer morphologically productive, the endings on \rz{s}-stems and \rz{r}-stems derive from historical derivation processes. Many roots are reflected in both endings, but the shared meaning between them is not always transparent.

\section{Deixis}
Nominal deixis has a variety of uses, including evidentiality, distance, familiarity, and topicality.\marginnote{I had this idea, then found out that, as usual, a natural language had it first. Read \href{http://lingpapers.sites.olt.ubc.ca/files/2020/07/11_ICSNL55_Huijsmans_Reisinger_Matthewson_final.pdf}{Huijsmans, Reisinger, and Matthewson (2020)} for more about the Salishan languages.} Verbs and adjectives exhibit agreement for deictic reference. There are three deictic categories, \emph{generic}, \emph{proximal}, and \emph{distal}.

\subsection{Generic}
The unmarked or dictionary form of a noun is used when the noun is widely understood or well-known, for immaterial referents that cannot be deictically located, or if evidence of the referent is not known. If direct or reported evidence exists, it's felicitous or questionably grammatical to use unmarked form.

\subsection{Proximal}
The proximal form of a noun is used when the speaker is certain, nearby, or familiar with the noun. It can also be used for the conversational topic. This form most commonly denotes direct evidence, meaning the speaker has personal experience with the marked noun. It is marked with the suffix \rz{-n}. \marginnote[*-2]{\rz{-n} is morphophonemically ⫽on⫽, where ⫽o⫽ doesn't surface for vocalic stems.}

% \begin{margintable}
%     \begin{tabular}{cc}
%         \toprule
%         \bf\sc gen & \bf\sc prox \\
%         \midrule
%         \rz{metka} & \rz{metkan} \\
%         \rz{retus} & \rz{ratusan} \\
%         \rz{pebar} & \rz{pabaran} \\
%         \bottomrule
%     \end{tabular}
%     \caption{Proximal inflection for different stems}
% \end{margintable}

\paragraph{Direct evidence}
The canonical meaning of the proximal form is direct evidence, often translated as “I saw.” 

\begin{kaobox}[frametitle=\sc todo:]
    The definiteness constructions probably need to be reworked to square better with (a) the stuff I've learned about definiteness and (b) the use of the deictic forms for topic/focus.
\end{kaobox}

\paragraph{Definiteness}
Proximal forms can be used to describe the definiteness of a referent. This construction is only used for weak, uniqueness-based definiteness (eg. “the Moon”), never for strong, anaphoric definiteness (eg. “the book”). For strong definitess, the noun \rz{sin} “???” is used alongside distal form, as in (\nextx b).

\begin{gloss*}
    \a \begingl
        \glpreamble Nassoin kąstecik su kagęsa su kagę́stapa. \endpreamble
            nassoi-n[king-\tsc{prox}]
            kąstecik[command]
            su[and]
            kagęsa[army]
            su[and]
            kagę́stapa[navy]
        \glft “The king (that we know) commands both army and navy.”
    \endgl
    \a \begingl
        \glpreamble sah ez-Rosąm pít ató sín. \endpreamble
            sah[soon]
            tę=rosąm[\tsc{prep}=cook]
            pít[hold\tbs\tsc{dist}]
            ató[grain\tbs\tsc{dist}]
            sín[???\tbs\tsc{dist}]
        \glft “The rice (that you mentioned) is about to be cooked.”
        \smoyd{https://www.reddit.com/r/conlangs/comments/kck1hi/1381st_just_used_5_minutes_of_your_day/}{1381}
    \endgl
\end{gloss*}

\subsection{Distal}
The distal form of a noun is used when the speaker is uncertain, far, or unfamiliar with the noun. It can also be used for the conversational focus. This form typically denotes indirect evidence, including inference, meaning the speaker has heard of or can make an educated guess about the existence of the marked noun. Reported deixis is marked by shifting stress to the ultimate syllable of the word.

\paragraph{Indirect Evidence}
The prototypical meaning of the distal form is indirect evidence, often translated as “heard about” or “they said.” As in (\nextx), this evidence is encoded into the clause via the subject and the predicate that agrees with it.

\begin{gloss*}
    \begingl
        \glpreamble egi Matkó aczé sém tę-het.\endpreamble
            egi[just]
            matkó[basket\tbs\tsc{dist}]
            aczé[two\tbs\tsc{dist}]
            sém[will:be\tbs\tsc{dist}]
            tę=het[\tsc{prep}=be:at]
        \glft “(She said) there will be just two baskets.”
        \smoyd{https://www.reddit.com/r/conlangs/comments/ic7on8/1314th_just_used_5_minutes_of_your_day/}{1314}
    \endgl
\end{gloss*}

\section{Number}
Nouns inflect morphologically for an additive plural, but there is also a syntactic construction used to form associative plurals. The unmarked form of a noun encodes expected number, \eg \rz{parsa} “eyes” which defaults to a pair and must take a numeral to specify a singulative. 

\begin{figure}[h]
    \centering
    \begin{subfigure}{0.4\textwidth}
        \centering
        \includegraphics[width=\textwidth]{same_vases.jpg}
        \caption{\rz{matkozar}}
    \end{subfigure}
    \begin{subfigure}{0.4\textwidth}
        \centering
        \includegraphics[width=\textwidth]{different_vases.jpg}
        \caption{\rz{ezzar metka}}
    \end{subfigure}
    \caption{Additive vs. associative plurals}
\end{figure}

The primary difference in the two plurals is the composition of the set: additive plurals refer to largely homogenous referents, whereas associative plurals refer to largely heterogenous referents.

\subsection{Additive plurals}
Additive plurals are used for a set of homogenous referents and never heterogenous referents; \eg \rz{matkozar} is “a set of the same (or similar) bowl” and never “a set of diverse bowls.”\marginnote{The second meaning would use the associative plural.}

Additive plurality is indicated through the suffix \rz{-zar}. Morphological marking is optional and a noun can be inferred additively plural from context. \marginnote{Mandatory plural marking is stylistically preferred in formal contexts.} As such marking is less common for small, discrete or easily countable sets or when a referent has been established plural in prior conversation. However, speakers are not always consistent with marking.

\begin{kaobox}[frametitle=\sc todo:]
    These tables probably belong in a different section (perhaps the overview in \emph{§ stems}), not here. Here can just re-hash the relevant bits of the inflection patterns.
\end{kaobox}

Morphonologically, plural marking precedes deictic marking; plural distal nouns have accent placed on the plural suffix, as seen in Table \ref{tab:metka_inflection}. The morphemes ⫽+n͡zaɹ+on⫽ are reduced to /ɹɹon/.

\rz{S}-stem and \rz{r}-stem nouns inflect similarly, except the proximal plural suffix is reduced to /n͡zzon/ instead.

Because the suffixes of \rz{s}-stems and \rz{r}-stems merge in the plural, some minimal pairs are rendered homphones when inflected. To combat this, speakers sometimes employ the use of the word \rz{tevi} “many” as a modifier for the singular form.

When number is specified with a numeral, the noun is not marked for plurality, as in (\nextx). This is another strategy to combat homophony.

\begin{gloss}
    \begingl
        \glpreamble vęci ocza \endpreamble
            vęci[mercenary]
            ocza[two]
        \glft “two mercenaries”
        \trailingcitation (cf. \rz{vącizar} “mercenaries”)
    \endgl
\end{gloss}

\subsection{Associative plurals}
Unlike the additive plural, the associative plural is not marked morphologically. The periphrastic construction \rz{...} conveys the associative meaning. 

\begin{kaobox}
Need to figure out the specific morpheme and/or construction. Could be \rz{ezzar}, \rz{ezzu}, a longer construction \dots
\end{kaobox}

The associative is used for a set of heterogenous referents. For animate (especially human) referents, the meaning is typically “a person and their associates,” as in (\nextx). The focal referent\marginnote{Terminology in this section adapted from \href{https://amor.cms.hu-berlin.de/~h2816i3x/Lehre/2007_VL_Typologie/03_Daniel_AssociativePlural.pdf}{Daniel and Moravcsik (2007)}.} (\ie most important) is the marked noun.

\begin{gloss}
    \begingl
        \glpreamble Sayanarnat otik ezzu Kanyi ez-laran. \endpreamble
            sayenar-nat[be.ignorant-\tsc{cvb}]
            ot-ik[be-\tsc{prox}]
            ezzu[\tsc{assoc}]
            Kanyi[\tsc{name}]
            ez=lar-n[\tsc{prep=expl-prox}]
        \glft “For this, Kanyi and his friends won't be much \emph{help}.”
    \endgl
\end{gloss}

The associative can also have a number of context-specific meanings, usually referring to diverse groups.

% \pex 
% \a \begingl
% \glpreamble ezzu Akakvazi rematt \endpreamble
% \glft “The diverse group of tag players entertained us ...”
% \endgl 
% % \a \begingl
% % \glft “My grandfather painted a wide variety of paintings.”
% % \endgl
% \xe

\section{Gender}
Some archaic nouns distinguish \emph{common} and \emph{neuter} gender, although this is largely a prescriptive convention. \marginnote{The gender distinction is more common in literary, academic, or scientific writings.} Loanwords, especially technical loanwords, are typically assigned neuter gender. Some words only distinguish gender for certain uses or contexts, thus dictionaries typically denote if a given usage is expected to require neuter gender.

\section{Pronouns}
Prounouns are morphologically and syntactically similar to nouns \dots

\begin{kaobox}[frametitle=\sc todo:]
Settle on pronominal forms---right now it's \rz{rat} \tsc{1}, \rz{a(f)} \tsc{2}, \rz{sec} \tsc{3c}, and \rz{moc} \tsc{3n}. There's some isogloss map about whether \tsc{2} is \rz{a} or \rz{af}.
\end{kaobox}

\setchapterpreamble[u]{\margintoc}
\chapter{Verbs}

\begin{kaobox}[frametitle=\sc todo:]
    Like in \emph{§ Nouns}, it would be good to have an overview of stems and inflection patterns before diving into the meaning of the morphemes.
\end{kaobox}

\section{Agreement}
Verbs agree with both the deictic position of the subject noun and the person of the two least oblique arguments of transitive verbs.

\subsection{Polypersonal}
Transitive verbs exhibit polypersonal agreement via a suffix to the verb root. Intransitive verbs don't require person marking, but it can be used for emphasis or clarification; in such cases, either the reflexive or \tsc{3c} patient morpheme is used.

%\begin{table}[h] \centering
%\begin{tabular}{cc|cccc}
%	& & \multicolumn{4}{c}{\textbf{Patient}} \\
%	& & \textbf{1} & \textbf{2} & \textbf{\tsc{3c}} & \textbf{\tsc{3n}} \\ \hline
%	\multirow{5}{*}{\textbf{Agent}} & \textbf{1} & - & c & s & n \\
%	& \textbf{2} & t & - & s & n \\
%	& \textbf{\tsc{3c}} & t & s & s & n \\
%	& \textbf{\tsc{3n}} & n & z & z & z \\
%	& \textbf{\tsc{refl}} & \multicolumn{4}{c}{k} \\
%\end{tabular}
%\caption{Person agreement}
%\end{table}

\begin{table}[h] \centering
\begin{tabular}{cc|cccc}
	& & \multicolumn{4}{c}{\textbf{Patient}} \\
	& & \textbf{1} & \textbf{2} & \textbf{\tsc{3c}} & \textbf{\tsc{3n}} \\ \midrule
	\multirow{5}{*}{\textbf{Agent}} & \textbf{1} & - & c\cellcolor{purple!25} & s\cellcolor{yellow!25} & n\cellcolor{blue!25} \\
	& \textbf{2} & t\cellcolor{green!25} & - & s\cellcolor{yellow!25} & n\cellcolor{blue!25} \\
	& \textbf{\tsc{3c}} & t\cellcolor{green!25} & s\cellcolor{yellow!25} & s\cellcolor{yellow!25} & n\cellcolor{blue!25} \\
	& \textbf{\tsc{3n}} & n\cellcolor{blue!25} & z\cellcolor{red!25} & z\cellcolor{red!25} & z\cellcolor{red!25} \\
	& \textbf{\tsc{refl}} & \multicolumn{4}{c}{k\cellcolor{black!25}} \\
\end{tabular}
\caption{Person agreement}
\end{table}

Because many agreement suffixes share the same form, \langname{} is only occasionally pro-drop.

\subsection{Deictic}
Generic noun forms do not have agreement morphemes, but proximal nouns demand the verbal suffix \rz{-ik} and distal nouns demand the verbal suffix \rz{-ǫ}. These suffixes are attached after polypersonal agreement suffixes.

\section{Negation}
Negation ...

The negative verb suffix is \rz{-res}. \marginnote{The affix's position is from its origin as an auxiliary which bore agreement.} It occurs before agreement affixes.

\section{Converb}
The converb form of a verb is used for simultaneous action. The converb is commonly used to describe the manner of the main clause, and is also commonly used in periphrastic constructions.

The converbial suffix is \rz{-nat}.

\section{Transitivity}
Transitivity is lexically fixed, but transitive verbs can still be functionally intransitive with the use of dummy objects.

Prescriptive convention holds that verbs exhibit polypersonal agreement with their dummy objects, but speakers commonly omit polypersonal marking in these constructions, as in (\nextx b).

\begin{gloss}
	\a \begingl
		\glpreamble sec Sasamsik tasa. \endpreamble
			sec[\tsc{3c}]
			sesam-s-ik[say-\tsc{3c.p-prox}]
			tasa[letter]
		\glft “He's saying something.”
		\trailingcitation (Formal)
	\endgl
	\a \begingl
		\glpreamble sec Semik lar. \endpreamble
			sec[\tsc{3c}]
			sem-ik[say-\tsc{prox}]
			lar[\tsc{exp}]
		\glft “He's talking.”
		\trailingcitation (Informal)
	\endgl \marginnote[*-3]{Informal constructions often use the more grammaticalized \rz{lar} instead of the dummy collocative.}
\end{gloss}

Not every verb has a collocated intransitive form. For these verbs, periphrastic constructions can also serve as valency-changing operations.

\section{Periphraxis}
Periphrastic constructions handle most temporal marking in \langname{}, covering aspects and moods. The core verb of the periphrastic construction is the only finite verb of a clause, bearing all agreement, and the semantic verb is demoted to an adjunct in converbial form or as a bare infinitive with an adpositional clitic.

\subsection{\rz{ot}}
The verb \rz{ot} “be” has two periphrastic constructions, a \emph{perfective} and a \emph{support} construction used for focus-fronting verbs.

\paragraph{Perfective}
In the perfective construction, the semantic verb is demoted to adjunct with the preposition \rz{tę}.

\paragraph{Support}
In the support construction, the semantic verb is demoted to adjunct as a converb.

\subsection{\rz{sesam}}
The verb \rz{sesam} “say” has one periphrastic construction, an irrealis.\marginnote{\rz{Sesam} is often shorted to \rz{sem} informally.}

\paragraph{Irrealis}
In the irrealis construction, the semantic verb is demoted to adjunct with the preposition \rz{tę}. 

\subsection{\rz{nenat}}
The verb \rz{nenat} “” has one periphrastic construction, a subjunctive.

\paragraph{Subjunctive}
In the subjunctive construction, the semantic verb is demoted to adjunct with the preposition \rz{ah}. The subjunctive construction has a more limited scope than the \rz{sesam} irrealis construction, typically expressing counterfactuals or doubt.

\subsection{\rz{het}}
The verb \rz{het} “be at” has two periphrastic constructions, a \emph{passive} and an \emph{imperfective}, the latter typically restricted to narrative contexts.

\paragraph{Passive}
In the passive construction, the semantic verb is demoted to adjunct as a converb. The \rz{het} passive is not a true passive because the verb does not change valency (\ie the \tsc{a}-like argument cannot be omitted). Instead of valency operations, the role of this construction is typically to change verbal agreement, as in (\nextx).

\begin{gloss*}
	\a \ljudge{*} \begingl
		\glpreamble Azzár hossusarsik nassoin.\endpreamble 
			as-zár[foreigner-\tsc{pl.dist}]
			hassusar-s-ik[exalt-\tsc{3c»3c-prox}]
			nassoi-n[king-\tsc{prox}]
		\glft \textit{Intended:} “(I see) the foreigners (I've heard about) praising the king.”
	\endgl
	\a \begingl
		\glpreamble Nassoin hecik azzár hossusarnat.\endpreamble
			nassoi-n[king-\tsc{prox}]
			het-s-ik[be.at\tsc{-3c»3c-prox}]
			as-zár[foreigner-\tsc{pl.dist}]
			hassusar-nat[exalt-\tsc{cvb}]
		\glft “(I see) the king being praised by the foreigners (I've heard about).”
	\endgl
\end{gloss*}

In (\lastx a), the hypothetical speaker intends to mark the verb as proximal to convey direct evidence, but the utterance is ungrammatical because the verb doesn't agree with its subject, \rz{azzár}. To correct this, the construction in (\lastx b) is used, which takes advantage of the passive to mark the verb phrase as proximal.

\par In addition to its role shuffling agreement, the \rz{het} passive can also be used to clarify sentences that become ambiguous due to focus fronting, as in (\nextx).

\begin{gloss*}
	\a \begingl
		\glpreamble Kanyin akakvatcik Arpatan. \endpreamble
			Kanyi-n[\tsc{name-prox}]
			akekvat-s-ik[tag-\tsc{3c»3c-prox}]
			Arpat-n[\tsc{name-prox}]
		\glft “Kanyi tagged Arpat.” \\ \textit{or} “Who Arpat tagged was Kanyi.”
	\endgl
	\a \begingl
		\glpreamble Kanyin hecik Arpatan akakvatnat. \endpreamble
			Kanyi-n[\tsc{name-prox}]
			het-s-ik[be.at\tsc{-3c»3c-prox}]
			Arpat-n[\tsc{name-prox}]
			akekvat-nat[tag-\tsc{cvb}]
		\glft “Kanyi was tagged by Arpat.”
	\endgl
\end{gloss*}

In (\lastx a), it's not clear if Kanyi is the semantic agent or a semantic patient that's been fronted for focus. Without context, both interpretations are grammatical. The use of the passive in (\lastx b) is less ambiguously interpreted, almost always meaning that Arpat was the semantic agent.

\subsection{\rz{pit}}
\begin{kaobox}[frametitle=\sc todo:]
    Rework the \rz{pit} passive, which currently doesn't make a lot of sense--how is it demoting stuff, when semantically you'd expect “hold” to be transitive? Maybe \rz{pit} means something else, maybe it will demote in a different way, maybe the valency is weirder...
\end{kaobox}
The verb \rz{pit} “hold” has one periphrastic construction, the mediopassive. \marginnote[*-2]{Unlike most other periphrastic verbs, \rz{pit} is rarely used outside periphrasis, having largely been replaced by \rz{akrar}.}

\paragraph{Mediopassive}
In the mediopassive construction, the semantic verb is demoted to adjunct with the preposition \rz{ez}. Unlike the \rz{het} passive, the \rz{pit} mediopassive is a true passive; the \tsc{a}-like argument does not appear.


\setchapterpreamble[u]{\margintoc}
\chapter{Adpositions}
Adpositions are syntactically bound morphemes that express some relationship (often spacial) between constituents. However, they are considered words, not affixes, because the stress pattern of the noun they bind to does not shift. \marginnote{Compare \rz{kąsazar} “soldiers,” marked via affix, to \rz{retus im-kęsa}, “soldier's blade,” marked via adposition.} Their phonological independence differentiates them from affixes.

Adpositions are a closed class, composed of only 5 members; finer distinctions can be made with periphrastic constructions, such as \rz{tę-kamc im} “after, to the back of.” Although many such constructions are common enough to be lexically set, they are not nearly as ubiquitous as lone prepositions.

\subsection{\textit{im}}
The adposition \rz{im} indicates possession.

\subsection{\textit{ez}}
The adposition \rz{ez} conveys location inside an object or large body.

\subsection{\textit{tę}}
The adposition \rz{tę} conveys motion relative to a location, either towards or away from.

\subsection{\textit{ah}}
The adposition \rz{ah} conveys location on the surface of another object.

\subsection{\textit{osc}}
The adposition \rz{osc} conveys location surrounding another object. It is commonly used in a temporal sense to indicate a time frame, often translated as “around the time of.”

\setchapterpreamble[u]{\margintoc}
\chapter{Adjectives}

\setchapterpreamble[u]{\margintoc}
\chapter{Particles}

\setchapterpreamble[u]{\margintoc}
\chapter{Clauses}
Although the base-generated word order in \langname{} is SVO, this order rarely surfaces due to aggressive focus fronting. The most proximal, most newsworthy information is placed in the front of an utterance in first position. As a consequence, \langname{} is V2 order, mandating that a finite verb always be in second position. Adjuncts, including demoted verbal constructions, typically come after the core arguments of the verb. \marginnote{In practice, the most common word order in declarative sentences is SXOV or VXOS.} Often, however, the order of elements in a clause is determined by focality and evidentiality.

\section{Fronting}
The most likely phrases to be fronted are proximal or directly evident noun phrases,\marginnote{Often, the fronted element will be the conversational focus.} followed by distal or indirectly evident noun phrases. Generic noun phrases are rarely fronted except in fixed constructions.

\langname{} doesn't have explicit role-marking, so it's not always clear if a fronted noun phrase is subject or object. Various structures, including verb agreement and passives, help provided redundancy when context is not enough.

\section{Extraposition}
When a content-heavy phrase needs to be fronted, a dummy noun is often used to allow right-branching extraposition. The dummy noun is lexically dependent, but the generic nouns \rz{lar} or \rz{manc} can also be used, although they may sound stilted for some verbs.

\begin{gloss*}
    \begingl
        \glpreamble osc Armę́ kirąyamik isyusan ocoan im nassoi kęstat ezzar vęci. \endpreamble
            osc[\tsc{prep}]
            armę́[\tsc{dmy}]
            kirąyamik[investigate]
            {isyusan ocoan}[special:tribune]
            im[\tsc{prep}]
            nassoi[king]
            kęstat[lead]
            ezzar[\tsc{assoc}]
            vęci[mercenary]
        \glft “What the special tribune is investigating is the king's use of mercenaries.”
    \endgl
\end{gloss*}

%\xe
%\begingl
%\glpreamble \endpreamble
%???[\tsc{expl}]
%risk-ǫ[\tsc{aux.neg-dist}]
%???[2]
%???[be.familiar]
%how-to[subj-clause]
%???-ya[friend-\tsc{prox}]
%???[go]
%???[to.here]
%???=im=tę[house=\tsc{poss=prep}]
%rat[1]
%\glft “It's someone you don't know, the friend that will come to our house.”
%\smoyd{https://www.reddit.com/r/conlangs/comments/ithoza/1330th_just_used_5_minutes_of_your_day/}{1330}
%\endgl
%\xe

% \pagelayout{plain}
% \part{Pragmatics}
% \pagelayout{margin}

\pagelayout{plain}
\part{Appendix}
\pagelayout{margin}

\appendix % From here onwards, chapters are numbered with letters, as is the appendix convention

% --- COMMANDS ---
\makeatletter
\define@key{entry}{nbm}[]{\def\entry@nbm{#1}}
\define@key{entry}{pos}[]{\def\entry@pos{#1}}
\define@key{entry}{etym}[]{\def\entry@etym{#1}}
\define@key{entry}{note}[]{\def\entry@note{#1}}

\setkeys{entry}{nbm,pos,etym,note}

\newcommand{\entry}[3][]{%
\begingroup%
\setcounter{sense}{1}%
\setkeys{entry}{#1}%
\par \begin{minipage}{\columnwidth}%
	% \rz{#2}%
	\rzbf{#2}%
	\enskip {\footnotesize \ifdefempty{\entry@nbm}{}{• \entry@nbm{} nbm.} \ifdefempty{\entry@pos}{}{• \entry@pos}} \\%
	\ifdefempty{\entry@note}{}{\enskip {\footnotesize \textit{note:} \entry@note} \\}%
	\ifdefempty{\entry@etym}{}{\enskip {\footnotesize ← from \entry@etym} \\}%
	#3\vspace{12pt}%
\end{minipage}%
\endgroup%
}

% linear (digital-style) definitions
%\newcounter{sense}
%\NewDocumentCommand{\sense}{o}{%
%\ifnum\the\value{sense}>1\\\fi%
%	\textbf{\arabic{sense}:}\enskip%
%	\IfNoValueTF{#1}{}{\textit{#1}}%
%\stepcounter{sense}%
%}

% cramped (print-style) definitions
\newcounter{sense}
\NewDocumentCommand{\sense}{o}{%
\ifnum\the\value{sense}>1\enskip\fi%
	\textbf{\arabic{sense}}\enskip%
	\IfNoValueTF{#1}{}{\textit{#1}}%
\stepcounter{sense}%
}

%\define@key{mean}{use}[]{\def\mean@use{#1}}
%\define@key{mean}{jrgn}[]{\def\mean@jrgn{#1}}
%
%\setkeys{mean}{use,jrgn}

%\newcommand{\sense}[1][]{%
%\begingroup%
%\setkeys{mean}{#1}%
%\ifnum\the\value{sense}>1\\\fi%
%	\textbf{\arabic{sense}:}\enskip%
%	\ifdefempty{\mean@jrgn}{}{\textsc{\mean@jrgn} \ifdefempty{\mean@use}{}{\enskip}}%
%	\ifdefempty{\mean@use}{}{\textit{\mean@use}}%
%\stepcounter{sense}%
%\endgroup%
%}

\newenvironment{dic}[1]{{\centering \addsec{#1}} \begin{multicols}{2}}{\end{multicols}}

\makeatother

\pagelayout{margin}
\setchapterpreamble[u]{\margintoc}
\chapter{Lexicon}
A \langname{}-to-English dictionary is provided below.

\section*{How to use}
Entries for lexical items are listed by their spelling in generic form, ignoring morphological alterations. Derived words are listed as separate entries, but their source word is given. On the other hand, idiomatic or fixed expressions are given under the lexical item.

Morphophonological information, such as \emph{nebami} vowels, underlying form, or irregular alterations, is listed when pertinent. Furthermore, senses that are specific to a certain word form are given in \textit{italics}, where as sense that are specific to a field or context are given in \textsc{\textit{small caps}}.

\section*{Examples}
Examples are generally given as simple declarative sentences, avoiding where possible movement due to focus fronting. Typically examples are chosen to provide context or illustrate usage notes for a definition.

\pagelayout{wide}
\setlength{\columnsep}{30pt}

% ===== A =====
\begin{dic}{A}
\entry[pos=n.,etym=Classical Cape \fz{dahēs} “pee”]{adahę́s}{\sense[uncountable] sand; cf. \emph{countable} \rz{sifa} “grain of sand” \sense[idiom.] citizens (of a nation, state), subjects, followers (of a leader, celebrity) \sense[\rz{ez-adahę́s}] the public, the people of a place: \rzit{Natra ez-adahę́s} “the Natran indigenous peoples”}
\entry[pos=n.,etym=Classical Cape \fz{gamrī}]{agąrę́}{\sense[\sc sailing] star}
\entry[pos=n.,etym=Classical Cape \fz{gamrka} “navigator”]{agątka}{\sense cartographer}
\entry[pos=n.]{ahka}{\sense 2 feet}
\entry[pos=v. tr.,etym=redup. of \rz{akvat}]{akakvat}{\sense (in a game) tag}
\entry[pos=n., etym=\rz{akakvat} + \rz{-zi}]{akakvazi}{\sense tagplayer (usually professional)}
\entry[pos=n.]{almani}{\sense wave \sense[idiom.] influence \sense[\sc politics] soft power}
\entry[pos=n.,etym=Classical Cape \fz{as} “man”]{as}{\sense foreigner}
\entry[pos=n.]{assoi}{\sense weeb for Classical Cape culture}
\end{dic}

% ===== Ą =====
\begin{dic}{Ą}
\entry[pos=n.,etym=\rz{ǫkas} “payment” + \rz{-suy}]{ąkassuy}{\sense bank}
\end{dic}

% ===== C =====
\begin{dic}{C}
\entry[pos=v. tr.]{cam}{\sense put (smn.) to sleep \sense[refl.] go to bed \sense[mediopassive] nap, snooze: \rzit{piran picik camnat tę-kamc im-pebar} “the child napped after school.” \sense[idiom.] calm (smn.) down, soothe \sense[coll.] bore (an audience): \rzit{ah-turya cad nakraran orran im-remaczi} “the comedian's subpar performance last night bored me.”}
\end{dic}

% ===== E =====
\begin{dic}{E}
\entry[pos=n.]{emas}{\sense[ntr.] plot of land}
\entry[pos=adj.]{esyi}{\sense good \sense correct, appropriate}
\end{dic}

% ===== Ę =====
\begin{dic}{Ę}
\entry[pos=v.,note=often \rz{ęrrat} for younger speakers]{ęrvat}{\sense }
\end{dic}

% ===== F =====

% ===== H =====
\begin{dic}{H}
\entry[pos=n.]{hakra}{\sense battle, skirmish \sense[pl.] conflict, campaign \sense[pl.] semester, trimester, school year}
\entry[pos=v. tr., etym=redup. of \rz{husar} “shout”]{hassusar}{\sense (esp. of a leader) exalt} % \sense[adj. → \rzit{hassusarem}] famous, well-known
\entry[pos=n.]{helas}{\sense[ntr.] mountain}
\entry[pos=n.]{hora}{\sense wrist}
\entry[pos=v. tr.]{husar}{\sense shout at \sense praise, compliment}
\end{dic}

% ===== I =====
\begin{dic}{I}
\entry[pos=n.]{ihaiha}{\sense donkey, mule \sense[adj. pej.] dumb}
\entry[pos=n.]{isyus}{\sense[ntr.] shield \sense[\textsc{culinary} ntr.] apron \sense[\textsc{military} pl.] elite, highly trained soldiers; royal guard, special forces, black ops: \rzit{pursę́ im-akąssuy ksarac isyuzzar} “the special forces respond to bank robberies.”}
\end{dic}

% ===== K =====
\begin{dic}{K}
\entry[pos=n.,note=underlying {/kamat͡s/,} often {[kams]} but {[kad͡z(ə)]} for some speakers]{kamc}{\sense back; the part of the body opposite the face below the neck and above the thigh, including the buttocks \sense[\rz{tę-kamc im}] after:}
\entry[pos=n.,etym=redup. of \rz{kęsa} “soldier”]{kagęsa}{\sense battalion, unit \sense (as a branch of the military) army}
\entry[pos=n.,etym=analogy with \rz{kagęsa} and \rz{kę́stapa}]{kagę́stapa}{\sense (as a branch of the military) navy}
\entry[pos=n.]{kęsa}{\sense soldier \sense[\sc academic] (of a literary work) protagonist, hero \sense[archaic] slave, conscript}
\entry[pos=n.,etym=\rz{kat} “pull” + \rz{-akz}]{katakz}{\sense[\rz{katakz im-saycezzar}] butterfly effect}
\entry[pos=n.,etym=\rz{kęsa} “soldier” + \rz{-suy}]{kąsasuy}{\sense high-ranking general \sense (with specifier) military officer: \rzit{kąsasuy ???} “field officer,” \rzit{kąsasuy ???} “medical officer”}
\entry[pos=n.,etym=\rz{kęsa} “soldier” and \rz{taspa} “sea”]{kę́stapa}{\sense navy officer, sailor (on a military ship)}
\entry[nbm=e,pos=v. tr.]{kęstat}{\sense lead \sense[\sc military] be in command of: \rzit{nassoin kąstecik su kagęsa su kagę́stapa} “the king commands both army and navy.” \sense train (an apprentice) in \rzit{tę} a skill: \rzit{tę-caradá kąstettik rat lalasa} “auntie's teaching me to sew.” \sense formally teach, school (a student) in \rzit{ez} a discipline: \rzit{rabahan picik kąstetnat ez-latya} “the minister was brought up in the faith.” \sense[pej.] indoctrinate}
\entry[pos=n.,etym=partial reduplication of \rz{pira} “child”]{kipira}{\sense adolescent; someone around or older than 8 years old who has not yet been ritually scarred}
\entry[pos=n.]{kotus}{\sense divinity}
\entry[pos=n.,etym=\rz{kotus} + \rz{-soi}]{kotussoi}{\sense body part; soul; the perfect implements wielded by an imperfect mind}
\entry[pos=v. in.]{kirǫyam}{\sense delve \sense descend deeper with forward motion into \rz{osc} some terrain (water, caves) to search for \rz{tę} something: \rz{tę-rasar kirąyamik osc-taspa} “he's swimming into the sea to search for rare fish.” \sense[\sc politics] conduct an investigation into \rz{osc} a topic: \rz{osc lar kirąyamik isyusan ocoan im nassoi kęstat ezzar vęci} “the special tribune is investigating the king's use of mercenary forces.” \sense[\rz{kirǫyam osc saycer}] wish for good luck for \rz{tę} someone: \rz{tę-af kirǫyam osc saycer tę-makra im-hakra} “good luck this semester!”}
\entry[pos=v. tr.]{ksarat}{\sense[\sc military] handle, respond to}
\entry[pos=n.]{ksofa}{\sense deciduous tree \sense growth, development (of the mind, socially)}
\end{dic}

% ===== L =====
\begin{dic}{L}
\entry[pos=n.]{lalasa}{\sense[\gsc{kinship}] great aunt \sense godmother \sense[affectionate] mentor}
\entry[pos=n.]{latya}{\sense religion}
\end{dic}

% ===== M =====
\begin{dic}{M}
\entry[pos=n.]{makra}{\sense chest; the part of the body below the neck and above the groin, including the shoulders and upper arms}
\entry[pos=n.]{mas}{\sense[ntr.] hour \sense[\rz{tę-kamc im-mas} adv.] in an hour: }
\entry[nbm=o,pos=n.]{metka}{\sense rounded semi-circle hollow container; bowl, basket, vessel  \sense measure word for crops or farm animals}
\end{dic}

% ===== N =====
\begin{dic}{N}
\entry[pos=n.,etym=\rz{nakrat} “” + \rz{-r}]{nakrar}{\sense performance (on stage)}
\entry[pos=n.,etym=\rz{naf} “gem” + \rz{-soi}]{nassoi}{\sense king}
\entry[pos=n.,etym=\rz{nassoi} “king” + \rz{-j} fossilized collective suffix,note=often {[nəsːojd͡z(ə)]}]{nassoij}{\sense aristocrat, bourgeois \sense[archaic] royal court, royal advisors}
\entry[pos=n.]{Natra}{\sense the mountain range that runs down the middle of the continent}
\entry[pos=n.,etym=Old East Cape]{nęcta}{\sense[\gsc{medical}] lung}
\entry[pos=n.]{nik}{\sense flat surface, plane \sense[adj.] flat}
\entry[pos=v. in.,nbm=o,etym=\rz{nik} “flat” + \rz{hor} “craft”]{nikhar}{\sense to create a map of \rzit{ez} a region}
\entry[pos=n.,etym=\rz{nikhar} “make maps” + \rz{rabaha} “minister”]{níkharkah}{\sense geography \sense[archaic] cartography}
\entry[pos=n.,etym=\rz{níkharkah} “geography” + \rz{-soi}]{nikhorkassoi}{\sense geographer \sense[archaic] cartographer}
\end{dic}

% ===== O =====
\begin{dic}{O}
\entry[pos=adj.,nbm=o,note=underlying /ot͡soa/]{oca}{\sense deep \sense (of people) tall and thin, wiry, spindly}
\entry[pos=n.,nbm=e]{ocza}{\sense two}
\entry[pos=n., nbm=o]{okva}{\sense valley}
\entry[pos=adj.]{orra}{\sense dry \sense (of food) stale \sense (of man-made objects) subpar, fragile}
\entry[pos=n.,nbm=o,etym=Classical Cape]{armę́}{\sense ocean}
\entry[pos=v. in.]{ossat}{\sense (of plants) to grow in size \sense (of people) to develop emotionally, intellectually}
\end{dic}

% ===== Ǫ =====
\begin{dic}{Ǫ}
\entry[pos=n.]{ǫkas}{\sense[ntr.] payment}
\end{dic}

% ===== P =====
\begin{dic}{P}
\entry[pos=n.]{pebar}{\sense garden \sense orchard \sense primary school}
\entry[pos=n.]{pira}{\sense child; someone around or under 8 years old who hasn't yet been brought fishing}
\entry[pos=n.]{polars}{\sense evergreen tree}
\entry[pos=n.,etym=Old Cape \fz{purcī} “plan”]{pursę́}{\sense robbery, heist}
\end{dic}

% ===== R =====
\begin{dic}{R}
\entry[pos=n.]{rabaha}{\sense[\sc religion] minister \sense[archaic] ministry}
\entry[pos=n.,nbm=e,etym=Old East Cape \fz{raxir} “fish”, note=sometimes spelled {\rz{raxir}, pronounced {[ɹaʃɛɹ]}, in elite circles or when exaggerating}]{rasar}{\sense meat (food) \sense[archaic] fish delicacy}
\entry[pos=n.]{recam}{\sense ritual scar made by a ceremonial blade on the upper arm near the shoulder of the dominant hand (historically on only the right arm) which symbolizes adulthood; usually done around the age of 16}
\entry[pos=n.,etym=\rz{recam} + \rz{-c}]{recaj}{\sense adulthood \sense the human condition, humanity, humanness}
\entry[pos=n.]{retus}{\sense[ntr.] blade \sense[\sc geography] canal}
\entry[pos=n., etym=\rz{remat} “” + \rz{-zi}]{remaczi}{\sense comedian}
\end{dic}

% ===== S =====
\begin{dic}{S}
\entry[pos=n.]{samni}{\sense[\sc geography] horn}
\entry[pos=n.,etym=Classical Cape \fz{sapa} “hunting dog”]{sapa}{\sense hunter}
\entry[pos=n.,etym=Old East Cape]{saska}{\sense[\gsc{medical}] heart}
\entry[pos=v. in.]{sayenar}{\sense be ignorant, uninformed, unknowledgeable about \rzit{ez} a topic: \rzit{ sayanarnat ez-latya tahątresik asan ah-mas esyi} “ignorant of the religion, the foreigner didn't pray at the correct hour.” }
\entry[pos=n., etym=\rz{secyat} “shine” + \rz{-r}]{secyar}{\sense celestial object: sun, moon, star \sense[\rz{sayceran im-turya}] the Moon \sense[\rz{saycer tea}] moon, satellite}
\entry[pos=n., etym=\rz{secyar} “star” + \rz{rabaha} “minister”]{sécyarkah}{\sense astronomy}
\entry[pos=n.,etym=\rz{sécyarkah} “astronomy” + \rz{-soi}]{sacyerkassoi}{\sense astronomer}
\entry[pos=v. tr.,note=often shortened to \rz{sem}]{sesam}{\sense say (smth.)}
\entry[pos=n.]{sifa}{\sense[countable] grain of sand \sense a tiny portion of \rzit{im} something \sense[name] girl given name}
\end{dic}

% ===== T =====
\begin{dic}{T}
\entry[pos=v. in.,nbm=ę]{tahąt}{\sense stretch (one's limbs) \sense reach for \rzit{???} something above oneself \sense aspire to \rzit{???} an unreachable goal: \rzit{... akakvazi ... oca ...} “he aspired to be a tagplayer, but he wasn't tall enough.” \sense[\sc religion] pray}
\entry[pos=n.]{taspa}{\sense sea}
\entry[pos=n.]{tǫsra}{\sense adult; one who has been ritually scarred with \rzit{recam}; someone around or older than 16}
\entry[pos=adj.]{tea}{\sense dark color: blue, black, violet}
\entry[pos=n.]{tesa}{\sense shore, shoreline (near the sand) \sense hairline}
\entry[pos=n.]{turya}{\sense night}
\end{dic}

% ===== U =====
\begin{dic}{U}
\entry[pos=n.,etym=unknown substrate language]{ukrak}{\sense shore near volcanic rock}
\end{dic}

% ===== V =====
\begin{dic}{V}
\entry[pos=n.]{vassa}{\sense tide \sense[idiom.] (of an event) historical, of great significance}
\entry[pos=n.]{vęci}{\sense mercenary, hired blade \sense[adj.] greedy \sense[pej.] brat, unruly child \sense[\rz{satat vęci} pej.] throw a temper tantrum: \rzit{vęci satacik piran ihaihan af} “your dumb kid is throwing a temper tantrum.”}
\end{dic}

% ===== Y =====
\begin{dic}{Y}
\entry[pos=n.,etym=\rz{yifat} “punish” + \rz{-r}]{yifar}{\sense robbery, theft (usually of small items)}
\end{dic}

% ===== Z =====
\begin{dic}{Z}
\entry[pos=n.,note=usually in proximal form]{zalmi}{\sense the Sun}
\end{dic}

\backmatter % Denotes the end of the main document content
\setchapterstyle{plain} % Output plain chapters from this point onwards
%\pagelayout{wide}

%\printbibliography[heading=bibintoc, title=References]

\end{document}
