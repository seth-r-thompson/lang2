\setchapterpreamble[u]{\margintoc}
\chapter{Adpositions}
Adpositions are syntactically bound morphemes that express some relationship (often spacial) between constituents. However, they are considered words, not affixes, because the stress pattern of the noun they bind to does not shift. \marginnote{Compare \rz{kąsazar} “soldiers,” marked via affix, to \rz{retus im-kęsa}, “soldier's blade,” marked via adposition.} Their phonological independence differentiates them from affixes.

Adpositions are a closed class, composed of only 5 members; finer distinctions can be made with periphrastic constructions, such as \rz{tę-kamc im} “after, to the back of.” Although many such constructions are common enough to be lexically set, they are not nearly as ubiquitous as lone prepositions.

\subsection{\rz{im}}
The adposition \rz{im} indicates possession.

\subsection{\rz{ez}}
The adposition \rz{ez} conveys location inside an object or large body.

\subsection{\rz{tę}}
The adposition \rz{tę} conveys motion relative to a location, either towards or away from.

\subsection{\rz{ah}}
The adposition \rz{ah} conveys location on the surface of another object.

\subsection{\rz{osc}}
The adposition \rz{osc} conveys location surrounding another object. It is commonly used in a temporal sense to indicate a time frame, often translated as “around the time of.”

\setchapterpreamble[u]{\margintoc}
\chapter{Adjectives}

\setchapterpreamble[u]{\margintoc}
\chapter{Particles}