\setchapterpreamble[u]{\margintoc}
\chapter{Adpositions}
Adpositions are syntactically bound morphemes that express some relationship (often spacial) between constituents. However, they are considered words, not affixes, because the stress pattern of the noun they bind to does not shift. \marginnote{Compare \rz{kąsazar} “soldiers,” marked via affix, to \rz{retus im-kęsa}, “soldier's blade,” marked via adposition.} Their phonological independence differentiates them from affixes.

Adpositions are a closed class, composed of only 6 members; finer distinctions can be made with \emph{compound adpositions}, such as \rz{tę-kamc im} “after, to the back of.” Although many such constructions are common enough to be lexically set, they are not nearly as ubiquitous as lone prepositions.

\subsection{\rz{im}}
The adposition \rz{im} indicates possession, often alienable. It can also indicate origin. 

\subsection{\rz{ez}}
The adposition \rz{ez} conveys location inside an object or large body. It can also be used for composition of manmade objects.

\subsection{\rz{tę}}
The adposition \rz{tę} conveys motion relative to a location, either towards or away from.

\subsection{\rz{ah}}
The adposition \rz{ah} conveys location on the surface of another object. It can also be used for general location.

\subsection{\rz{osc}}
The adposition \rz{osc} conveys location surrounding another object. It is commonly used in a temporal sense to indicate a time frame, often translated as “around the time of.”

\subsection{\rz{u} and \rz{su}} \label{subsec:u_and_su}
\rz{U} and \rz{su} are more limited in semantic scope than other adpositions and are rarely used in compound prepositions to gain further nuance. However, they have similar syntactic distribution and are thus considered members of the preopsition class. They typically convey association alongisde.

While \rz{u} is used largely to join two noun phrases as arguments of one head, such as \rz{laran u lár} “this and that,” \rz{su} is used for emphasize or in some fixed constructions, such as \rz{su kagęsa su kagę́stapa} “both army and navy.”

\paragraph{Stranded \rz{su}}
Unlike other prepositions, \rz{su} can appear \emph{stranded}, \ie\ without modifying. For example, while (\nextx a) violates the valency of the verb, (\nextx b) does not because the prepositional phrase functions as the argument of the verb.

\begin{gloss*}
    \a \ljudge{*} \begingl
        \glpreamble Nassoin kąstecik tę-kagęsa. \endpreamble
        nassoin[king]
        kąstecik[commands]
        tę=kagęsa[\tsc{prep}=army]
        \glft \emph{Intended:} “The king commands the army.”
    \endgl
    \a \begingl
        \glpreamble Nassoin kąstecik su kagęsa su kagę́stapa. \endpreamble
        nassoin[king]
        kąstecik[commands]
        su[and]
        kagęsa[army]
        su[and]
        kagę́stapa[navy]
        \glft “The king commands both army and navy.”
    \endgl
\end{gloss*}

\begin{kaobox}[frametitle=\sc todo:]
    This “stranded” analysis is probably really sketchy, since it's clearly not fulfilling the role of other prepositions, but I don't feel like opening that can of worms just yet so take it at face value for now.
\end{kaobox}

\setchapterpreamble[u]{\margintoc}
\chapter{Adjectives}
Adjectives are a small, closed class of noun-like morphemes that cannot be arguments of verbs or prepositions without some other constituent. There are approximately 20 adjectives. Adjectives agree with their head noun for deixis (but not number) and syntactically appear before appositives or prepositional modifiers in the noun phrase.

\setchapterpreamble[u]{\margintoc}
\chapter{Particles}
Particles are a small but open class of discourse markers that can appear at the beginning of a clause. The most common particles are \rz{???}, a polar question marker, \rz{???}, a content question marker, and \rz{ǫm}, an imperative marker. They are syntactically privileged, able to occur before other constituents of a clause, including even fronted arguments or adjuncts.

\section{Imperative}
The imperative particle \rz{ǫm} \dots

For prohibitives, the most common construction is to negate the verb, as in (\nextx).

\begin{gloss}
    \begingl 
        \glpreamble ǫm af Itnatréks! \endpreamble
        ǫm[\tsc{imp}]
        af[2]
        itnet-réks[fight-\tsc{neg.prox}]
        \glft “Don't \emph{you} fight me!”
        \smoyd{https://www.reddit.com/r/conlangs/comments/mejb4o/1440th_just_used_5_minutes_of_your_day/gsk44w4?utm_source=share&utm_medium=web2x&context=3}{1440}
    \endgl
\end{gloss}

However, constructions with \rz{kers} are also common, especially in the southwest.\marginnote{East Cape uses similar constructions for its prohibitives, likely influencing the increased modal force of \rz{kers}} \rz{Kers} has a more admonitive meaning in standard \langname{}, often used in warnings or chidings, but wouldn't be used for commands or requests. 

\begin{figure}[h]
    \includegraphics[width=\linewidth]{isogloss_prohibitives.png}
    \caption{Isogloss map of prohibitives}
\end{figure}