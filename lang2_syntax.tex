\setchapterpreamble[u]{\margintoc}
\chapter{Syntax}
\section{Clause}
Although the base-generated word order in \langname{} is SVO, this order rarely surfaces due to aggressive focus fronting. The most proximal, most newsworthy information is placed in the front of an utterance in first position. As a consequence, \langname{} is V2 order, mandating that a finite verb always be in second position. Adjuncts, including demoted verbal constructions, typically come after the core arguments of the verb. \marginnote{In practice, the most common word order in declarative sentences is SXOV or VXOS.} Often, however, the order of elements in a clause is determined by focality and evidentiality.

\subsection{Fronting}
The most likely phrases to be fronted are proximal or directly evident noun phrases,\marginnote{Often, the fronted element will be the conversational focus.} followed by distal or indirectly evident noun phrases. Generic noun phrases are rarely fronted except in fixed constructions.

\subsection{Extraposition}
When a content-heavy phrase needs to be fronted, a dummy noun is often used to allow right-branching extraposition. The dummy noun is lexically dependent, but the generic nouns \rz{lar} or \rz{manc} can also be used, although they may sound stilted.

\begin{gloss*}
    \begingl
    \glpreamble osc Armę́ kirąyamik isyusan ocoan im nassoi kęstat ezzar vęci. \endpreamble
    osc[\gsc{prep}]
    armę́[\gsc{dmy}]
    kirąyamik[investigate]
    {isyusan ocoan}[special:tribune]
    im[\gsc{prep}]
    nassoi[king]
    kęstat[lead]
    ezzar[\gsc{assoc}]
    vęci[mercenary]
    \glft “What the special tribune is investigating is the king's use of mercenaries.”
    \endgl
\end{gloss*}

%\xe
%\begingl
%\glpreamble \endpreamble
%???[\gsc{expl}]
%risk-ǫ[\gsc{aux.neg-dist}]
%???[2]
%???[be.familiar]
%how-to[subj-clause]
%???-ya[friend-\gsc{prox}]
%???[go]
%???[to.here]
%???=im=tę[house=\gsc{poss=prep}]
%rat[1]
%\glft “It's someone you don't know, the friend that will come to our house.”
%\smoyd{https://www.reddit.com/r/conlangs/comments/ithoza/1330th_just_used_5_minutes_of_your_day/}{1330}
%\endgl
%\xe

\section{Phrase}
Phrasal elements are typically head first, so nouns precede their modifiers and verbs precede their oblique arguments and adjuncts.